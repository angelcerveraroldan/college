\documentclass[12pt]{article} % use the article class

\usepackage[plain]{algorithm} % algorithms package
\usepackage{algpseudocode} % pseudo-code package
\usepackage{amsbsy} % for producing bold maths symbols
\usepackage{amsfonts} % an extended set of fonts for maths
\usepackage{amssymb} % various maths symbols
\usepackage{amsthm} % for producing theorem-like environments
\usepackage{datetime2} % managing dates and times
\usepackage{delimseasy} % makes easy the manual sizing of brackets, square brackets, and curly brackets
\usepackage{enumitem} % customing list environments
\usepackage{extramarks} % extra marks
\usepackage{fancyhdr} % headers and footers
\usepackage{float} % makes dealing with floats (e.g. tables and figures) easier
\usepackage{framed} % for producing framed boxes
\usepackage{graphicx} % for including graphics in the document
\usepackage{hyperref} % automatically produce hyperlinks for cross-references
\usepackage{import} % Import and subimport
\usepackage{listings} % Code blocks
\usepackage{mathtools} % package for maths (fixes some deficiences of amsmath so is preferred)
\usepackage{mdframed} % boxes
\usepackage{microtype} % better font sizing (extremely helpful with long equations!)
\usepackage{newtx} % a fonts package
\usepackage{parskip}
\usepackage{pdfpages} % for including pdf documents inside the compiled pdf
\usepackage{pgf} % produce pdf graphics using LaTeX
\usepackage{pgfplots} % create normal/logarithmic plots in two and three dimensions
\pgfplotsset{compat=1.18} % sorts out the compatability warning
\usepackage{physics} % useful for vector calculus and linear algebra symbols
\usepackage[most]{tcolorbox} % for producing coloured boxes
\tcbuselibrary{theorems} % theorems with tcolorbox
\usepackage{tikz-3dplot} % for producing 3d plots
\usepackage{tikz} % for drawing graphics in LaTeX
\usepackage{tkz-base} % drawing with a Cartesian coordinate system
\usepackage{tkz-euclide} % drawing in Euclidean geometry
\usepackage{xcolor} % a package for colours


%%% DOCUMENT SETTINGS
\topmargin=-0.45in
\evensidemargin=0in
\oddsidemargin=0in
\textwidth=6.5in
\textheight=9.0in
\headsep=0.25in
\linespread{1.1}
\pagestyle{fancy}
\lhead{\hmwkAuthorName}
\chead{}
% \rhead{\hmwkStudentnum}
\rhead{\MakeLowercase{\leftmark}}
\lfoot{\lastxmark}
\cfoot{\thepage}
\renewcommand\headrulewidth{0.4pt}
\renewcommand\footrulewidth{0.4pt}
\setlength\parindent{0pt}


\renewcommand{\part}[1]{\textbf{\large Part \Alph{partCounter}}\stepcounter{partCounter}\\} % part macro
\newcommand{\solution}{\textbf{\large Solution}} % solution macro

\newmdenv[
    backgroundcolor=gray!20,
    skipabove=\topsep,
    skipbelow=\topsep,
]{grayBoxed}

% General writing
\newcommand{\bracket}[1]{\left(#1\right)} % for automatic resizing of brackets
\newcommand{\sbracket}[1]{\left[#1\right]} % for automatic resizing of square brackets
\newcommand{\mset}[1]{\left\{#1\right\}} % for automatic resizing of curly brackets
\newcommand{\defeq}{\coloneqq} % the "defined as" command
\newcommand{\RNum}[1]{\uppercase\expandafter{\romannumeral #1\relax}} % uppercase roman numerals

\newcommand{\rednote}[1]{{\color{red} #1}} % for red text
\newcommand{\bluenote}[1]{{\color{blue} #1}} % for blue text
\newcommand{\greennote}[1]{{\color{green} #1}} % for green text

% Blackboard Maths Symbols
\newcommand{\N}{\mathbb{N}} % natural numbers
\newcommand{\Q}{\mathbb{Q}} % rational numbers
\newcommand{\Z}{\mathbb{Z}} % integers
\newcommand{\R}{\mathbb{R}} % real numbers
\newcommand{\C}{\mathbb{C}} % complex numbers
\newcommand{\E}{\mathbb{E}} % expectation operators

% Aesthetic
\newcommand{\thmBox}[2]{
    \begin{tcolorbox}[colback=blue!5!white,colframe=blue!50!black,
            colbacktitle=blue!90!black,title=Theorem #1]
        #2
    \end{tcolorbox}
}

\newcommand{\lmBox}[2]{
    \begin{tcolorbox}[colback=blue!5!white,colframe=blue!50!black,
            colbacktitle=blue!50!black,title=Lemma #1]
        #2
    \end{tcolorbox}
}

\newcommand{\defBox}[2]{
    \begin{tcolorbox}[colback=red!5!white,colframe=red!50!black,
            colbacktitle=red!60!black,title=Definition #1]
        #2
    \end{tcolorbox}
}


\newcommand{\exerciseBox}[2]{
    \begin{tcolorbox}[colback=red!5!white,colframe=red!50!black,
            colbacktitle=red!60!black,title=Definition #1]
        #2
    \end{tcolorbox}
}

% Homework Specific
\newcommand{\hBox}[1]{
    \begin{tcolorbox}[colback=blue!5!white,colframe=blue!50!black]
        #1
    \end{tcolorbox}
}

\newcommand{\hmwkStudentnum}{21319203}
\newcommand{\hmwkAuthorName}{\textbf{Angel Cervera Roldan}}

%%% Define the homeworkProblem environment
\newcommand{\enterProblemHeader}[1]{
    \nobreak\extramarks{}{Problem \arabic{#1} continued on next page\ldots}\nobreak{}
    \nobreak\extramarks{Problem \arabic{#1} (continued)}{Problem \arabic{#1} continued on next page\ldots}\nobreak{}
}

\newcommand{\exitProblemHeader}[1]{
    \nobreak\extramarks{Problem \arabic{#1} (continued)}{Problem \arabic{#1} continued on next page\ldots}\nobreak{}
    \stepcounter{#1}
    \nobreak\extramarks{Problem \arabic{#1}}{}\nobreak{}
}

\setcounter{secnumdepth}{0}
\newcounter{partCounter}
\newcounter{homeworkProblemCounter}
\setcounter{homeworkProblemCounter}{1}
\nobreak\extramarks{Problem \arabic{homeworkProblemCounter}}{}\nobreak{}

\newenvironment{homeworkProblem}[1][-1]{
    \ifnum#1>0
        \setcounter{homeworkProblemCounter}{#1}
    \fi
    \section{Problem \arabic{homeworkProblemCounter}}
    \setcounter{partCounter}{1}
    \enterProblemHeader{homeworkProblemCounter}
}{
    \exitProblemHeader{homeworkProblemCounter}
}
 
\newcommand{\classTitle}{Class Name}
\newcommand*{\halfpi}{\frac{\pi}{2}}

\newcommand\optim[2]{\xrightarrow{(#1)\times r_#2}}
\newcommand\opadd[3]{\xrightarrow{r_#1{}+{} #2r_#3}}

\rhead{Abstract algebra assignment 1}

\title{
    \vspace{2in}
        \textmd{\textbf{\classTitle}}\\
    \vspace{1in}
    \textmd{\textbf{Homework \#2}}\\
    \vspace{1in}
}

\author{
    \hmwkAuthorName\\
    \hmwkStudentnum\\
}

\date{}

\begin{document}

\maketitle

\pagebreak


% Problem 1
\begin{homeworkProblem}

    \hBox{
        Find the basis of the following subspace of $\R^4$:

        $$
            S_1 := \mset{(x_1, x_2, x_3, x_4) \in \R^4 \mid  x_1 - 2x_2 + x_4 = 0 , x_3 + x_4 = 0}
        $$

        $$
            S_2 := span(\mset{
                \begin{pmatrix} 1 \\ 0 \\ 1 \\ 0 \end{pmatrix},
                \begin{pmatrix} 5 \\ 6 \\ 5 \\ -6 \end{pmatrix},
                \begin{pmatrix} 1 \\ 2 \\ 1 \\ -2 \end{pmatrix},
                \begin{pmatrix} 3 \\ -4 \\ 3 \\ 4 \end{pmatrix}
            })
        $$

        Also, find the dimension of $S_1 \cap S_2$
    }


    \subsection*{Part 1}

    \begin{align*}
        x_1 - 2x_2 + x_4 & = 0 \implies x_4 = 2x_2 - x_1 \\
        x_3 + x_4        & = 0 \implies x_3 = - x_4      \\
    \end{align*}

    And so, given $(x_1, x_2, x_3, x_4)$ can be written as $(x_1, x_2, -(2x_2 - x_1), 2x_2 - x_1)$

    Therefore

    $$
        S_1 = span(\mset{
            \begin{pmatrix} 1 \\ 0 \\ 0 \\ 0  \end{pmatrix},
            \begin{pmatrix} 0 \\ 1 \\ 0 \\ 0  \end{pmatrix},
            \begin{pmatrix} 0 \\ 0 \\ 1 \\ -1 \end{pmatrix}
        })
    $$

    meaning that a basis for $S_1$ is

    $$
        \mset{
            \begin{pmatrix} 1 \\ 0 \\ 0 \\ 0  \end{pmatrix},
            \begin{pmatrix} 0 \\ 1 \\ 0 \\ 0  \end{pmatrix},
            \begin{pmatrix} 0 \\ 0 \\ 1 \\ -1 \end{pmatrix}
        }
    $$

    \pagebreak

    \subsection*{Part 2}

    To find the basis for $S_2$, we can put the given vectors into a matrix, and simplify it to Reduced Row Echelon Form:


    \begin{align*}
        \begin{bmatrix}
            1 & 0  & 1 & 0  \\
            5 & 6  & 5 & -6 \\
            1 & 2  & 1 & -2 \\
            3 & -4 & 3 & 4  \\
        \end{bmatrix}
        \xrightarrow{r_2 - 5r_1}
         & \begin{bmatrix}
               1 & 0  & 1 & 0  \\
               0 & 6  & 0 & -6 \\
               1 & 2  & 1 & -2 \\
               3 & -4 & 3 & 4  \\
           \end{bmatrix} \\
        \xrightarrow{r_3 - r_1}
         & \begin{bmatrix}
               1 & 0  & 1 & 0  \\
               0 & 6  & 0 & -6 \\
               0 & 2  & 0 & -2 \\
               3 & -4 & 3 & 4  \\
           \end{bmatrix} \\
        \xrightarrow{r_4 - 3r_1}
         & \begin{bmatrix}
               1 & 0  & 1 & 0  \\
               0 & 6  & 0 & -6 \\
               0 & 2  & 0 & -2 \\
               0 & -4 & 0 & 4  \\
           \end{bmatrix} \\
        \xrightarrow{1/6 r_2 \text{, }  1/2 r_3 \text{, } -1/4r_4}
         & \begin{bmatrix}
               1 & 0 & 1 & 0  \\
               0 & 1 & 0 & -1 \\
               0 & 1 & 0 & -1 \\
               0 & 1 & 0 & -1 \\
           \end{bmatrix}  \\
        \xrightarrow{r_3 - r_2 \text{, } r_4 - r_2}
         & \begin{bmatrix}
               1 & 0 & 1 & 0  \\
               0 & 1 & 0 & -1 \\
               0 & 0 & 0 & 0  \\
               0 & 0 & 0 & 0  \\
           \end{bmatrix}  \\
    \end{align*}

    Therefore:

    $$
        S_2 = span(\mset{
            \begin{pmatrix} 1 \\ 0 \\ 1 \\ 0 \end{pmatrix},
            \begin{pmatrix} 0 \\ 1 \\ 0 \\ -1 \end{pmatrix}
        })
    $$

    meaning that a basis for $S_2$ is

    $$
        \mset{
            \begin{pmatrix} 1 \\ 0 \\ 1 \\ 0 \end{pmatrix},
            \begin{pmatrix} 0 \\ 1 \\ 0 \\ -1 \end{pmatrix}
        }
    $$

    \subsection*{Part 3}

    Any element in $S_1 \cap S_2$ must be in $S_1$ and in $S_2$, therefore, $x \in S_2$ can be written as $(a, b, a, -b)$, also, $x \in S_1 \therefore x = (n, m, c, -c)$
    This means that $a = -(-b) = b$, therefore, $x = (b, b, b, -b)$.

    Therefore
    $$
        S_1 \cap S_2 = span(\mset{
            \begin{pmatrix} 1 \\ 1 \\ 1 \\ -1 \end{pmatrix}
        })
    $$

    And so, the dimension of $S_1 \cap S_2$ is 1.
\end{homeworkProblem}
\pagebreak


% Problem 2
\begin{homeworkProblem}
    \hBox {
        $$
            S := \mset{P \in \R_n[X] \mid P(1) = 0}
        $$

        Show that $S$ is a subspace of $\R_n[X]$, find a basis for $S$, and determine its dimension.
    }

    Given $A, B \in S$, and $\lambda \in \R$, to show that $S$ is a subspace of $\R_n[X]$, we need to prove:

    \begin{enumerate}
        \item $(A + B)(x) = A(x) + B(x)$
        \item $(\lambda A)(x) = \lambda A(x)$
    \end{enumerate}

    By the definition of the polynomials, $\forall P, Q \in \R_n[X]$, $(P + Q)(x) = P(x) + Q(x)$, and $\forall \alpha \in \R$, $(\alpha P)(x) = \alpha P(x)$

    $$
        (A + B)(1) = A(1) + B(1) = 0 + 0 = 0
    $$


    $$
        (\lambda A)(x) = \lambda A(x) = \lambda 0 = 0
    $$

    And so this shows that for any $A, B \in S$, $A + B \in S$, and for any $\lambda \in \R$, $\lambda A \in S$, therefore, $S$ is a subspace of $\R_n[X]$.

    The basis of $\R_n[X]$ is given by $\mset{1, x, x^2, ..., x^n}$. However, since $A(1) = 0$, we know that:

    $$
        A = a_0 + a_1 x^1 + a_2 x^2 + ... + a_n x^n
    $$

    $$
        A(1) = a_0 + a_2 + ... + a_n = 0 \implies a_0 = - (a_1 + ... + a_n)
    $$

    Therefore, $A$ can be written as $a_1(x - 1) + a_2(x^2 - 1) + ... + a_n(x^n - 1) = (a_1 x - a_1) + ... + (a_n x^n - a_n)$

    This shows that $A = span(\mset{x - 1, x^2 - 1, ..., x^n - 1})$, therefore, $dim(S) = n$
\end{homeworkProblem}
\pagebreak

% Problem 3
\begin{homeworkProblem}

    \hBox{
        Show that

        $$
            M_{n \times n}(\R) = Sym_{n}(\R) \oplus Skew_{n}(\R)
        $$
    }

    To prove the above, we need to show that every $A \in M_{n \times n}(\R)$ can be written as a unique sum of $B + C$, $B \in Sym_{n}(\R)$, $C \in Skew_{n}(\R)$

    Note: By the definition, $b_{ij} = b_{ji}$, and $c_{ij} = - c_{ji}$ this also shows that $c_{ii} = 0$ since $c_{ii} = -c_{ii}$.

    We can find the matrices $A$ and $B$ algebraically

    \begin{align*}
        a_{ij} & = b_{ij} + c_{ij}                   \\
        a_{ji} & = b_{ji} + c_{ji} = b_{ij} - c_{ij} \\
    \end{align*}

    Since we need to solve for 2 variables ($b_{ij}$, and $c_{ij}$) with two given values ($a_{ij}$, and $a_{ji}$), there is only one unique solution.

    \begin{align*}
        b_{ij} & = a_{ji} + c_{ij}                                                       \\
        a_{ij} & = a_{ji} + c_{ij} + c_{ij} = a_{ji} + 2 c_{ij}                          \\
        c_{ij} & = \frac{a_{ij} - a_{ji}}{2}, \quad c_{ji} = - \frac{a_{ij} - a_{ji}}{2} \\
        b_{ij} & = b_{ji} = a_{ji} + c_{ij} = a_{ji} + \frac{a_{ij} - a_{ji}}{2}
    \end{align*}

    Since every element in $B$ and every element in $C$ can be written in one way in terms of the elements of $A$, we know that the sum is unique.

\end{homeworkProblem}
\pagebreak

% Problem 4
\begin{homeworkProblem}
    \hBox{
        Which of the following is a linear functional

        \begin{enumerate}
            \item on vector space $\R[X]$ over $\R$: $f(P) = 3P(0) + 5P(1)$
            \item on vector space $\R^3$ over $\R$: $f(x_1, x_2, x_3) = x_1x_2 + x_2x_3 + x_3x_1$
            \item on vector space $\R^3$ over $\R$: $f(x_1, x_2, x_3) = \abs{x_1} + \abs{x_2} + \abs{x_3}$
            \item on vector space $\Z_5^3$ over $\Z_5$: $f(x_1, x_2, x_3) = x_1^5 + x_2^5 + x_3^5$
        \end{enumerate}
    }

    Definition: A function $f$ is linear functional iff:

    $$
        f(a + b) = f(a) + f(b)
    $$

    $$
        f(\lambda a) = \lambda f(a)
    $$

    \subsection*{Part 1}

    Take $P, Q \in \R[X]$, and $\lambda \in \R$, then:

    \begin{align*}
        f(P + Q)
         & = 3(P + Q)(0) + 5(P + Q)(1)         \\
         & = 3(P(0) + Q(0)) + 5(P(1) + Q(1))   \\
         & = 3P(0) + 3Q(0) + 5P(1) + 5Q(1)     \\
         & = 3P(0) + 5P(1) + 3Q(0) + 5Q(1)     \\
         & = f(P) +f(Q)                        \\ \\
        f(\lambda P)
         & = 3(\lambda P)(0) + 5(\lambda Q)(1) \\
         & = 3\lambda P(0) + 5 \lambda Q(1)    \\
         & = \lambda (3P(0) + 5Q(1))           \\
         & = \lambda f(P)                      \\
    \end{align*}

    This shows that the function $f(P) = 3P(0) + 5P(1)$ is linear functional


    \subsection*{Part 2}

    We can prove that $f(x_1, x_2, x_3) = x_1x_2 + x_2x_3 + x_3x_1$ is not linear functional by contradiction:


    If $f$ were linear functional, then $- f(x_1, x_2, x_3) = f(-x_1, -x_2, -x_3)$, however:

    $$
        f(-x_1, -x_2, -x_3)  = (-x_1)(-x_2) + (-x_2)(-x_3) + (-x_3)(-x_1) = x_1 x_2 + x_2 x_ 3 + x_3 x_1 = f(x_1, x_2, x_3)
    $$

    And since $f(-x_1, -x_2, -x_3) = f(x_1, x_2, x_3)$, then $f(-x_1, -x_2, -x_3) \not = - f(x_1, x_2, x_3)$, meaning that $f$ is not linear functional.

    \subsection*{Part 3}

    We can show $f(x_1, x_2, x_3) = \abs{x_1} + \abs{x_2} + \abs{x_3}$ is not linear functional by example.

    $f(1, 1, 1) = 3$, if $f$ were linear functional, then $- f(1, 1, 1) = f(-1, -1, -1) = -3$, however, $f(-1, -1, -1) = 3 \not = -3$.

    Therefore,$f(\lambda x) \not = \lambda f(x)$ for all $x$, which means that $f$ is not linear functional.

    \subsection*{Part 4}

    $f(x_1, x_2, x_3) = x_1^5 + x_2^5 + x_3^5$ is linear functional.

    \begin{align*}
        f(\lambda x_1, \lambda x_2, \lambda x_3)
         & = (\lambda x_1)^5 + (\lambda x_2)^5 + (\lambda x_3)^5 &  & \mod{5} \\
         & = \lambda^5 x_1^5 + \lambda^5 x_2^5 + \lambda^5 x_3^5 &  & \mod{5} \\
         & = \lambda^5 (x_1^5 + x_2^5 + x_3^5)                   &  & \mod{5} \\
         & = \lambda (x_1^5 + x_2^5 + x_3^5)                     &  & \mod{5} \\
    \end{align*}

    Note: $(\lambda^5 = \lambda) \mod{5}$ we can prove this by showing all possible values for $\lambda \in \Z_5$

    \begin{align*}
        1^5 & = 1 \mod{5}        &  & 2^5 = 32 = 2 \mod{5}   \\
        3^5 & = 243 = 3 \mod{5}  &  & 4^5 = 1024 = 4 \mod{5} \\
        5^5 & = 3125 = 0 \mod{5}
    \end{align*}

    Note: $(a + b)^5 = a^5 + 5 a^4 b + 10 a^3 b^2 + 10 a^2 b^3 + 5 a b^4 + b^5 = a^5 + b^5 \mod{5}$

    \begin{align*}
        f(x_1 + y_1, x_2 + y_2, x_3  + y_3)
         & = (x_1 + y_1)^5 + (x_2 + y_2)^5 + (x_3  + y_3)^5 &  & \mod{5} \\
         & = x_1^5 + y_1^5 + x_2^5 + y_2^5 + x_3^5 + y_3^5  &  & \mod{5} \\
         & = x_1^5 + x_2^5 + x_3^5 + y_1^5 + y_2^5 + y_3^5  &  & \mod{5} \\
         & = f(x_1, x_2, x_3) + f(y_1, y_2, y_3)
    \end{align*}

\end{homeworkProblem}

\pagebreak

% Problem 5
\begin{homeworkProblem}

    \subsection*{Part 1}
    Poof by contradiction:

    $S$ is linearly dependent if $\exists \lambda$'s $\in \Q$ (not all 0), st:

    $$
        \sum_{i = 1}^{\infty} \lambda_i log(p_i) = 0
    $$

    (assume that those lambdas exist)

    Note: $log(x) = 0 \implies x = 1$

    \begin{align*}
        \sum_{i = 1}^{\infty} \lambda_i log(p_i)
         & =\sum_{i = 1}^{\infty} log(p_i^{\lambda_i})  \\
         & = log(\sum_{i = 1}^{\infty} p_i^{\lambda_i}) \\
         & = 0                                          \\
        \sum_{i = 1}^{\infty} p_i^{\lambda_i}
         & = 1
    \end{align*}

    But for the sum to be equal to one, $\lambda_i = 0$ for all values of $i$, this shows that $S$ is not linearly dependent, meaning that it is linearly independent.

    \subsection*{Part 2}

    We showed that $S$ is linearly independent, therefore, if we assume that $\R_\Q$ is finite dimensional, with a dimension of $n$, then
    the first $n$ elements of $S$ would be a basis for $\R_\Q$.

    However, the $n + 1$ element of $S$ is also an element of $\R_\Q$, and it cannot be written in terms of the firs $n$ elements that we chose (because S is linearly independent),
    therefore, those elements cannot be a basis.

    This is a contradiction, showing that $\R_\Q$ is not finite dimensional.

\end{homeworkProblem}
\pagebreak

% Problem 6
\begin{homeworkProblem}

    \hBox{
        Let $V$ be a vector space over a field $\F$, and suppose that
        $f, g \in V^*$ are such that $$g(x) = 0 \iff f(x) = 0$$ %% Should this be a one way implicatoin ?
    }

    Take $x_0 \in V$ such that $g(x_0) \not = 0$ (if no such $x_0$ exists, then $g(x) = 0$ for all x, and therefore, $f = g$, so $\lambda = 1$). Define $n := g(x_0)$, and $m := f(x_0)$.
    We can rewrite $m$ in terms of $n$: $m = (mn^{-1}) n$, let $k = m n^{-1} \in \F$. Therefore, $n = g(x_0)$, $kn = f(x_0)$.

    First, we need to show that $\forall x \in V$, $\exists \alpha$ such that $g(x + \alpha x_0) = 0$ we can do so algebraically:

    Since $g$ is linear, $g(0) = 0$ (*), therefore, find $\alpha$ such that $x - \alpha x_0 = 0 \implies \alpha = x x_0^{-1}$

    \begin{align*}
        g(x - \alpha x_0)    & = 0             \\
        g(x) - \alpha g(x_0) & =  0            \\
        g(x)                 & = \alpha g(x_0) \\
                             & = \alpha n      \\ \\
        f(x - \alpha x_0)    & =0              \\
        f(x) - \alpha f(x_0) & =0              \\
        f(x)                 & = \alpha f(x_0) \\
                             & = \alpha kn     \\
                             & = k \alpha n    \\
                             & = k g(x)        \\
    \end{align*}

    Let $\lambda = k$, and we have proven that there exists $\lambda$ such that $f = \lambda g$


    (*) Take $t$ to be a linear functional, then $f(0) = 0$. This can be shown as follows: $f(x + 0) = n = f(x)$, $f(x) + f(0) = n$
\end{homeworkProblem}

\end{document}
