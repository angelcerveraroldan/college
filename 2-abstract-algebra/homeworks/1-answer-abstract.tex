\documentclass[12pt]{article} % use the article class

\usepackage[plain]{algorithm} % algorithms package
\usepackage{algpseudocode} % pseudo-code package
\usepackage{amsbsy} % for producing bold maths symbols
\usepackage{amsfonts} % an extended set of fonts for maths
\usepackage{amssymb} % various maths symbols
\usepackage{amsthm} % for producing theorem-like environments
\usepackage{datetime2} % managing dates and times
\usepackage{delimseasy} % makes easy the manual sizing of brackets, square brackets, and curly brackets
\usepackage{enumitem} % customing list environments
\usepackage{extramarks} % extra marks
\usepackage{fancyhdr} % headers and footers
\usepackage{float} % makes dealing with floats (e.g. tables and figures) easier
\usepackage{framed} % for producing framed boxes
\usepackage{graphicx} % for including graphics in the document
\usepackage{hyperref} % automatically produce hyperlinks for cross-references
\usepackage{import} % Import and subimport
\usepackage{listings} % Code blocks
\usepackage{mathtools} % package for maths (fixes some deficiences of amsmath so is preferred)
\usepackage{mdframed} % boxes
\usepackage{microtype} % better font sizing (extremely helpful with long equations!)
\usepackage{newtx} % a fonts package
\usepackage{parskip}
\usepackage{pdfpages} % for including pdf documents inside the compiled pdf
\usepackage{pgf} % produce pdf graphics using LaTeX
\usepackage{pgfplots} % create normal/logarithmic plots in two and three dimensions
\pgfplotsset{compat=1.18} % sorts out the compatability warning
\usepackage{physics} % useful for vector calculus and linear algebra symbols
\usepackage[most]{tcolorbox} % for producing coloured boxes
\tcbuselibrary{theorems} % theorems with tcolorbox
\usepackage{tikz-3dplot} % for producing 3d plots
\usepackage{tikz} % for drawing graphics in LaTeX
\usepackage{tkz-base} % drawing with a Cartesian coordinate system
\usepackage{tkz-euclide} % drawing in Euclidean geometry
\usepackage{xcolor} % a package for colours


%%% DOCUMENT SETTINGS
\topmargin=-0.45in
\evensidemargin=0in
\oddsidemargin=0in
\textwidth=6.5in
\textheight=9.0in
\headsep=0.25in
\linespread{1.1}
\pagestyle{fancy}
\lhead{\hmwkAuthorName}
\chead{}
% \rhead{\hmwkStudentnum}
\rhead{\MakeLowercase{\leftmark}}
\lfoot{\lastxmark}
\cfoot{\thepage}
\renewcommand\headrulewidth{0.4pt}
\renewcommand\footrulewidth{0.4pt}
\setlength\parindent{0pt}


\renewcommand{\part}[1]{\textbf{\large Part \Alph{partCounter}}\stepcounter{partCounter}\\} % part macro
\newcommand{\solution}{\textbf{\large Solution}} % solution macro

\newmdenv[
    backgroundcolor=gray!20,
    skipabove=\topsep,
    skipbelow=\topsep,
]{grayBoxed}

% General writing
\newcommand{\bracket}[1]{\left(#1\right)} % for automatic resizing of brackets
\newcommand{\sbracket}[1]{\left[#1\right]} % for automatic resizing of square brackets
\newcommand{\mset}[1]{\left\{#1\right\}} % for automatic resizing of curly brackets
\newcommand{\defeq}{\coloneqq} % the "defined as" command
\newcommand{\RNum}[1]{\uppercase\expandafter{\romannumeral #1\relax}} % uppercase roman numerals

\newcommand{\rednote}[1]{{\color{red} #1}} % for red text
\newcommand{\bluenote}[1]{{\color{blue} #1}} % for blue text
\newcommand{\greennote}[1]{{\color{green} #1}} % for green text

% Blackboard Maths Symbols
\newcommand{\N}{\mathbb{N}} % natural numbers
\newcommand{\Q}{\mathbb{Q}} % rational numbers
\newcommand{\Z}{\mathbb{Z}} % integers
\newcommand{\R}{\mathbb{R}} % real numbers
\newcommand{\C}{\mathbb{C}} % complex numbers
\newcommand{\E}{\mathbb{E}} % expectation operators

% Aesthetic
\newcommand{\thmBox}[2]{
    \begin{tcolorbox}[colback=blue!5!white,colframe=blue!50!black,
            colbacktitle=blue!90!black,title=Theorem #1]
        #2
    \end{tcolorbox}
}

\newcommand{\lmBox}[2]{
    \begin{tcolorbox}[colback=blue!5!white,colframe=blue!50!black,
            colbacktitle=blue!50!black,title=Lemma #1]
        #2
    \end{tcolorbox}
}

\newcommand{\defBox}[2]{
    \begin{tcolorbox}[colback=red!5!white,colframe=red!50!black,
            colbacktitle=red!60!black,title=Definition #1]
        #2
    \end{tcolorbox}
}


\newcommand{\exerciseBox}[2]{
    \begin{tcolorbox}[colback=red!5!white,colframe=red!50!black,
            colbacktitle=red!60!black,title=Definition #1]
        #2
    \end{tcolorbox}
}

% Homework Specific
\newcommand{\hBox}[1]{
    \begin{tcolorbox}[colback=blue!5!white,colframe=blue!50!black]
        #1
    \end{tcolorbox}
}

\newcommand{\hmwkStudentnum}{21319203}
\newcommand{\hmwkAuthorName}{\textbf{Angel Cervera Roldan}}

%%% Define the homeworkProblem environment
\newcommand{\enterProblemHeader}[1]{
    \nobreak\extramarks{}{Problem \arabic{#1} continued on next page\ldots}\nobreak{}
    \nobreak\extramarks{Problem \arabic{#1} (continued)}{Problem \arabic{#1} continued on next page\ldots}\nobreak{}
}

\newcommand{\exitProblemHeader}[1]{
    \nobreak\extramarks{Problem \arabic{#1} (continued)}{Problem \arabic{#1} continued on next page\ldots}\nobreak{}
    \stepcounter{#1}
    \nobreak\extramarks{Problem \arabic{#1}}{}\nobreak{}
}

\setcounter{secnumdepth}{0}
\newcounter{partCounter}
\newcounter{homeworkProblemCounter}
\setcounter{homeworkProblemCounter}{1}
\nobreak\extramarks{Problem \arabic{homeworkProblemCounter}}{}\nobreak{}

\newenvironment{homeworkProblem}[1][-1]{
    \ifnum#1>0
        \setcounter{homeworkProblemCounter}{#1}
    \fi
    \section{Problem \arabic{homeworkProblemCounter}}
    \setcounter{partCounter}{1}
    \enterProblemHeader{homeworkProblemCounter}
}{
    \exitProblemHeader{homeworkProblemCounter}
}
 
\newcommand{\classTitle}{Class Name}

\newcommand*{\halfpi}{\frac{\pi}{2}}

\rhead{Abstract algebra assignment 1}

\title{
    \vspace{2in}
        \textmd{\textbf{\classTitle}}\\
    \vspace{1in}
    \textmd{\textbf{Homework \#1}}\\
    \vspace{1in}
}

\author{
    \hmwkAuthorName\\
    \hmwkStudentnum\\
}

\date{}

\begin{document}

\maketitle

\pagebreak

% Problem 1 
\begin{homeworkProblem}

    \hBox{
        Consider $\mathbb{F} := \mset{(x_1, x_2) \mid x_1, x_2 \in \Q}$ with the following operations:

        \begin{align*}
            (x_1, x_2) + (y_1,y_2)     & := (x_1 + y_1, x_2 + y_2)                  \\
            (x_1, x_2) \cdot (y_1,y_2) & := (x_1 y_1 + 5x_2 y_2, x_1 y_2 + x_2 y_1)
        \end{align*}


        \begin{enumerate}
            \item What is the multiplicative identity of $\F$
            \item Given $(x_1, x_2) \not = (0, 0)$, find $(x_1, x_2)^{-1}$
        \end{enumerate}
    }

    \subsection*{Part 1}

    Claim: $(1, 0)$ is the multiplicative identity of $\F$.

    To prove this, we need to show 2 things:

    \begin{enumerate}
        \item $(1, 0) \in \F$, we know this holds since $1, 0 \in \Q$.
        \item $\forall (x_1, x_2) \in \F$,  $(x_1, x_2) \cdot (1, 0) = (x_1, x_2)$, this can be shown algebraically:
    \end{enumerate}


    \begin{align*}
        (x_1, x_2) \cdot (1, 0)
         & = (x_1 \cdot (1) + 5x_2 \cdot (0), x_1 \cdot (0) + x_2 \cdot (1)) \\
         & = (x_1, x_2)                                                      \\
    \end{align*}

    \subsection*{Part 2}

    Take $(x_1, x_2) \in \F$, by definition, if $(a, b)$ is the multiplicative inverse of $(x_1, x_2)$, then

    $$
        (x_1, x_2) \cdot (a, b) = e_1 = (1, 0)
    $$

    We can solve for $a, b$ algebraically:

    \begin{align*}
        (x_1, x_2) \cdot (a, b)
               & =(1, 0)                                                   \\
        (x_1, x_2) \cdot (a, b)
               & = (x_1 \cdot a + 5x_2 \cdot b, x_1 \cdot b + x_2 \cdot a) \\
        (1, 0) & = (x_1 \cdot a + 5x_2 \cdot b, x_1 \cdot b + x_2 \cdot a) \\
    \end{align*}

    From that, we get the following two equations:

    \begin{align}
        1 & = x_1 \cdot a + 5x_2 \cdot b \\
        0 & = x_1 \cdot b + x_2 \cdot a
    \end{align}

    From equation 2, we get

    \begin{equation}
        a = \frac{- x_1 \cdot b}{x_2} = - \frac{x_1}{x_2} \cdot b
    \end{equation}


    Now, we can plug in the value for $a$ we found in equation 3 into equation 1:


    $$
        1 = x_1 \cdot a + 5x_2 \cdot b = x_1 \cdot (- \frac{x_1}{x_2} \cdot b) + 5x_2 \cdot b = - \frac{x_1^2}{x_2} \cdot b + 5 x_2 \cdot b = b (-\frac{x_1^2}{x_2} + 5x_2)
    $$

    If we multiply both sides of the equation by $x_2$,


    $$
        x_2 = b (-x_1^2 + 5x_2^2)
    $$

    From that, we can isolate $b$ in terms of $x_1$ and $x_2$.

    $$
        b = \frac{x_2}{5x_2^2 - x_1^2}
    $$

    We can now find a by subbing the value of b we just found into equation 3:

    $$
        a = - \frac{x_1}{x_2} \cdot b = - \frac{x_1}{x_2} \cdot \frac{x_2}{5x_2^2 - x_1^2} = - \frac{x_1}{5x_2^2 - x_1^2}
    $$

    The inverse would be undefined when $5x_2^2 - x_1^2 = 0$, however $5x_2^2 - x_1^2$ cannot be 0 if $x_1, x_2 \in \Q - \{(0)\}$. We can show this by contradiction, assume $x_1, x_2 \in \Q-\{(0)\}$, and $5x_2^2 - x_1^2 = 0$

    $$
        5x_2^2 - x_1^2 = 0 \implies 5x_2^2 = x_1^2 \implies \sqrt{5}x_2 = x_1
    $$

    However, $x_1, x_2 \in \Q$, therefore $\sqrt{5}x_2 \not \in \Q$, and so $5x_2^2 - x_1^2$ can not be 0.

    The inverse is undefined when $5x_2^2 - x_1^2 = 0$, therefore, the inverse is only undefined at $(0, 0)$, and any other value will have an inverse.

    Now, we can verify that the inverse actually works:

    \begin{align*}
        (x_1, x_2) \cdot (a, b)
         & = (x_1, x_2) \cdot (- \frac{x_1}{5x_2^2 - x_1^2}, \frac{x_2}{5x_2^2 - x_1^2})                                                                                     \\
         & = (x_1 \cdot (- \frac{x_1}{5x_2^2 - x_1^2}) + 5 x_2 \frac{x_2}{5x_2^2 - x_1^2},  x_1 \cdot \frac{x_2}{5x_2^2 - x_1^2} + x_2 \cdot (- \frac{x_1}{5x_2^2 - x_1^2})) \\
         & = (-\frac{x_1^2}{5 x_2^2 - x_1^1} + \frac{5x_2^2}{5x_2^2 - x_1^2}, \frac{x_1 x_2}{5x_2^2 - x_1^2} - \frac{x_2 x_1}{5x_2^2 - x_1^2})                               \\
         & = (\frac{5x_2^2 - x_1^2}{5 x_2^2 - x_1^1}, 0)                                                                                                                     \\
         & = (1, 0)
    \end{align*}

\end{homeworkProblem}
\pagebreak













% Problem 2
\begin{homeworkProblem}

    \hBox{
        Let $S := \R^2$, with the following operations

        \begin{align*}
            \boxplus & \rightarrowtail (x_1, x_2) \boxplus (y_1, y_2) = (x_1 + y_2, x_2 + y_1) \\
            \cdot    & \rightarrowtail \lambda  (x_1, x_2) =  (\lambda x_1, \lambda x_2)
        \end{align*}

        Is $(S, \R, \boxplus, \cdot)$ a vector space?
    }

    It is not a vector space, as $\boxplus$ is not commutative, we can show this by counter example:

    Take $(1, 3), (3, 4) \in S$. If $S$ is a vector space, then: $(1, 3) \boxplus (3, 4) = (3, 4) \boxplus (1, 3)$, however:

    $$
        (1, 3) \boxplus (3, 4) = (1 + 4, 3 + 3) = (5, 6)
    $$

    $$
        (3, 4) \boxplus (1, 3) = (3 + 3, 4 + 1) = (6, 5)
    $$

    Therefore, $(1, 3) \boxplus (3, 4) \not = (3, 4) \boxplus (1, 3)$, and so commutativity does not hold.

\end{homeworkProblem}
\pagebreak













% Problem 3
\begin{homeworkProblem}
    \hBox{
        Which of the following are subspaces of $\R^3$

        \begin{itemize}
            \item $S = \mset{(x_1, x_2, x_3) \in \R^3 \mid 2x_1 - x_2 = 0}$
            \item $S = \mset{(x_1, x_2, x_3) \in \R^3 \mid 2x_1 - x_2^2 = 0}$
            \item $S = \mset{(x_1, x_2, x_3) \in \R^3 \mid 2x_1 - x_2 > 0}$
        \end{itemize}
    }

    \subsection*{Part 1}
    $$
        S = \mset{(x_1, x_2, x_3) \in \R^3 \mid 2x_1 - x_2 = 0}
    $$

    This is a subspace of $\R^3$

    Take $(x_1, x_2, x_3), (y_1, y_2, y_3) \in S$, and $\lambda \in \R$, we need to show that $(x_1, x_2, x_3) + (y_1, y_2, y_3) \in S$, and that $\lambda (x_1, x_2, x_3) \in S$.


    \begin{align*}
        2x_1 - x_2                  & = 0 \\
        2y_1 - y_2                  & = 0 \\
        2x_1 - x_2 + 2y_1 - y_2     & = 0 \\
        2 (x_1 + y_1) - (x_2 + y_2) & = 0 \\
    \end{align*}

    Therefore, $(x_1 + y_1, x_2 + y_2, x_3 + y_3) \in S$

    \begin{align*}
        2x_1 - x_2                       & = 0 \\
        \lambda \cdot (2x_1 - x_2)       & = 0 \\
        2( \lambda x_1 ) - (\lambda x_2) & = 0 \\
    \end{align*}

    Therefore, $(\lambda x_1, \lambda x_2, \lambda x_3) \in S$

    This shows that $S$ is a subspace of $\R^3$.

    \subsection*{Part 2}
    $$
        S = \mset{(x_1, x_2, x_3) \in \R^3 \mid 2x_1 - x_2^2 = 0}
    $$

    This is not a subspace.

    Take $(x_1, x_2, x_3) \in S \& \lambda \in \R$, then, if $S$ is a subspace, $\lambda (x_1, x_2, x_3) \in S$, however, we can find a counter example.

    Take $(2, 2, 0) \in \R^3$, this is also an element of $S$ since $2(2) - 2^2 = 4 - 4 = 0$. Take $\lambda = 2$.

    Then, if $S$ is a subspace $2 \cdot (2, 2, 0) = (4, 4, 0) \in S$. However, $(4, 4, 0) \not \in S$ because $2(4) - 4^2 = 8 - 16 = -8 \not = 0$.



    \subsection*{Part 3}
    $$
        S = \mset{(x_1, x_2, x_3) \in \R^3 \mid 2x_1 - x_2 > 0}
    $$

    This is not a subspace, as is it not closed under multiplication. To prove it, we can assume that $S$ is a subspace, and find a counter example.

    Take $(x_1, x_2, x_3) \in S$, we assumed that $S$ is a subspace, therefore, $\forall \lambda \in \R \quad \lambda (x_1, x_2, x_3) \in S$.

    Take $\lambda = -1$, then, $-1 (x_1, x_2, x_3) = (-x_1, -x_2, -x_3) \in S$, however, if $2x_1 - x_2 > 0$, then

    $2(-x_1) - (-x_2) \not> 0$, and so $(-x_1, -x_2, -x_3) \not \in S$.




\end{homeworkProblem}
\pagebreak














% Problem 4
\begin{homeworkProblem}
    \hBox{
        Show that if $V$ is a finite-dimensional vector space and $U$ is a subspace of $U$, then it also is finite dimensional.
    }

    We know that $V$ is finite dimensional. Let $n = dim(V)$. From definition 6.4, we have that any list of $n$ elements in $V$ that is linearly dependent is a basis for $V$.

    We can prove that $U$ is finite dimensional by contradiction. To do this, let's first assume that $U$ is not finite dimensional. This means that there doesn't exist a finite list of linearly
    independent elements of $U$ that is a basis for $U$.

    Take the list $u_1 \in U$, this is a linearly dependent list. Because $U$ is not finite dimensional, and so $u_1$ cannot be a basis for it, there must exist an element $u_2 \in U$ st $u_2 \not \in \langle u_1 \rangle$.

    Now add $u_2$ to the list, $u_1, u_2 \in U$ must still be linearly independent, since $u_2 \not \in \langle u_1 \rangle$. We can repeat the above logic, because $U$ is not finite dimensional,
    then $u_1, u_2$ cannot be a basis for it, and there must exist $u_3 \in U$ st $u_3 \not \in \langle u_1, u_2 \rangle$ ....

    Repeat that process until we have a list $u_1, ..., u_n \in U$ that is linearly independent. Here we will find the contradiction:

    From definition 6.4 as stated above, $u_1, ..., u_n \in U$ must be a basis for $V$ since it is linearly independent, and every element is in $V$ (because $U \subseteq V$). This means
    that $\forall u \in U$, $u \in \langle u_1, ..., u_2 \rangle$.

    However, because $U$ is not finite dimensional, there must exist $u_{n + 1} \in U$ st $u_{n + 1} \not \in \langle u_1, ..., u_2 \rangle$.

    This is a contradiction, meaning that $u_{n+1}$ cannot exist, and so $U$ cannot be finite dimensional, and can have at most a dimension of $n$.


\end{homeworkProblem}
\pagebreak














% Problem 5
\begin{homeworkProblem}

    \hBox{
        Given that $T, U, W$ are subspaces of some vector space $V$, prove that the following is false:

        \begin{align*}
            dim(T + U + W)
            = & dim(T) + dim(U) + dim(W)               \\
              & - dim(T \cap U)- dim(U \cap W)         \\
              & - dim(W \cap T) + dim(T \cap U \cap W)
        \end{align*}
    }

    We can prove that this is false by counter example, take $V = \R^3$, and

    \begin{align*}
        T & := \mset{(x, 0, 0) \in \R^3} = \langle(1, 0, 0)\rangle \\
        U & := \mset{(0, x, 0) \in \R^3} = \langle(0, 1, 0)\rangle \\
        W & := \mset{(x, x, 0) \in \R^3} = \langle(1, 1, 0)\rangle \\
    \end{align*}

    Since $T, U, $ and $W$ all have a basis of just one element, $dim(T) = 1$, $dim(U) = 1$, $dim(W) = 1$.


    $T \cap U = \mset{(0, 0, 0)}$, therefore, $dim(T \cap U) = 0$.

    $U \cap W = \mset{(0, 0, 0)}$, therefore, $dim(U \cap W) = 0$.

    $W \cap T = \mset{(0, 0, 0)}$, therefore, $dim(W \cap T) = 0$.

    $T \cap U \cap W = \mset{(0, 0, 0)}$, therefore, $dim(T \cap U \cap W) = 0$

    By the equation given, we would expect $dim(T + U + W) = 1 + 1 + 1 - 0 - 0 - 0 + 0 = 3$


    However,

    $$T + U + W = \mset{t + u + w \mid t \in T \& u \in U \& w \in W}$$

    or

    $$\mset{(x, y, 0) \in \R^3}$$


    Therefore, $T + U + W = \langle (1, 0, 0), (0, 1, 0) \rangle$, meaning that $dim(T + U + W) = 2$.

    Given that the given equation gave us the dimension to be 3, but the actual dimension is 2, this shows that the equation given is false.

\end{homeworkProblem}

\end{document}
