\section{Lecture Feb 28 2023}


\defBox{}{
    Take $V, W$ to be vector spaces over $\F$. Take $T, S$

    \begin{align*}
        T: V \to W \\
        S: V \to W
    \end{align*}

    And define $T + S$ to be:

    $$
        (T + S)(v) = T(v) + S(v)
    $$

    and, for any $\lambda \in \F$

    $$
        (\lambda T)(v) = \lambda T(v)
    $$
}


\exerciseBox{}{
    Show that T + S and lambda T are linear maps V to W
}

Let's call

$$
    L(V, W) = \mset{T: V \to W \st \text{ T is linear }}
$$

Then, $(L(V, W), \F, +, \cdot)$ is a vector space.

Another operation:

\begin{align*}
    T: V \to W \\
    S: W \to U
\end{align*}

We can define the composition of $T$ and $S$ to be $(T \circ S)(v) = S(T(v))$.

\textbf{Claim}: $S, T \text{ being linear } \implies S \circ T \text{ is linear }$

\begin{align*}
    S(T(\lambda v + \mu w))
     & = S(\lambda T(v) + \mu T(w))                  \\
     & = \lambda S(T(v)) + \mu S(T(w))               \\
     & = \lambda (S \circ T)(v) + \mu (S \circ T)(w)
\end{align*}

If $V = W$, we write $L(V, V) = L(V)$

\section*{Properties of composition}
\begin{enumerate}
    \item $R \circ (S \circ T) = (R \circ S) \circ T$
    \item $R \circ (S + T) = R \circ S + R \circ T$
    \item $S \circ (\lambda \cdot T) = \lambda \cdot (S \circ T) = (\lambda \cdot S) \circ T$
\end{enumerate}

Side note: All of these properties make $L(v)$ an \textbf{algebra}

\section*{Invertible linear maps}

\defBox{Identity Map}{
    The identity map $I$ is defined by
    $$
        I(v) = v
    $$

    Where $I \in L(V)$
}

We can see that $\forall T \in L(V)$, $T \circ I = T = I \circ T$

\defBox{}{
    A linear map $T \in L(V)$ is invertible iff $\exists S \in L(V)$ such that:

    $$
        S \circ T = I = T \circ S
    $$
}

\thmBox{The right inverse and the left inverse are always the same}{
    Suppose that $R, S \in L(v)$, if $ST = I$, and $TR = I$, then $S = R$
}

To prove this, we can see the following:

\begin{align*}
    (ST)R
     & = STR   \\
     & = S(TR) \\
     & = SI    \\
     & = S     \\
    (ST)R
     & = IR    \\
     & = R
\end{align*}

Therefore, $S = R$

This also leads us to believe that if $T$ is inversible, then its inverse is unique.
To show this, we can assume that there exist two different inverses of $T$, $S_1, S_2$.

Then,

\begin{align*}
    S_1 T = I = T S_1  \\
    S_2 T = I = T S_2  \\
    S_1 T = I = T S_ 2 \\
    \implies S_1 = S_2
\end{align*}

Since the inverse is unique, we can denote it by $T^{-1}$.


\lmBox{}{
    $V, W$ are vector spaces over $\F$, $T \in L(V, W)$

    $$
        T \text{ is injective } \Longleftrightarrow (T(v) = 0 \implies v = 0)
    $$
}

To prove this, assume that $T$ is injective. Then, if $v \not = 0$,

$$
    T(v) \not = T(0) = 0
$$

Assume that $T(v) = 0 \Leftrightarrow v = 0$. Suppose that there are $v, u \in V$ such that
$$
    T(v) = T(u)
$$

By linearity, we have
$$
    0 = T(v) - T(u) =  T(v - u) \implies v - u = 0 \text{ meaning that } v = u
$$

\lmBox{}{
    $T \in L(V)$ is injective iff
    \begin{enumerate}
        \item T is injective
        \item T is surjective
    \end{enumerate}
}

To prove this, we can assume that $T$ s a bijection. Then we define

$$
    T^{-1} (w) = \text{ the unique $v \in V$ st $T(v) = w$}
$$

We need to show that $T{^1}$ is linear.

$$
    T^{-1}(\lambda_1 w_1 + \lambda_2 w_2) = u \st T(u) = \lambda_1 w_1 + \lambda_2 w_2 \\
$$

\begin{align*}
    u =  \lambda_1 w_1 + \lambda_2 w_2
\end{align*}

% TODO  finish from pics



Now we need to prove it the other way around.