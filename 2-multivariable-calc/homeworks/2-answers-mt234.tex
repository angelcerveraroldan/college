\documentclass[12pt]{article} % use the article class

\usepackage[plain]{algorithm} % algorithms package
\usepackage{algpseudocode} % pseudo-code package
\usepackage{amsbsy} % for producing bold maths symbols
\usepackage{amsfonts} % an extended set of fonts for maths
\usepackage{amssymb} % various maths symbols
\usepackage{amsthm} % for producing theorem-like environments
\usepackage{datetime2} % managing dates and times
\usepackage{delimseasy} % makes easy the manual sizing of brackets, square brackets, and curly brackets
\usepackage{enumitem} % customing list environments
\usepackage{extramarks} % extra marks
\usepackage{fancyhdr} % headers and footers
\usepackage{float} % makes dealing with floats (e.g. tables and figures) easier
\usepackage{framed} % for producing framed boxes
\usepackage{graphicx} % for including graphics in the document
\usepackage{hyperref} % automatically produce hyperlinks for cross-references
\usepackage{import} % Import and subimport
\usepackage{listings} % Code blocks
\usepackage{mathtools} % package for maths (fixes some deficiences of amsmath so is preferred)
\usepackage{mdframed} % boxes
\usepackage{microtype} % better font sizing (extremely helpful with long equations!)
\usepackage{newtx} % a fonts package
\usepackage{parskip}
\usepackage{pdfpages} % for including pdf documents inside the compiled pdf
\usepackage{pgf} % produce pdf graphics using LaTeX
\usepackage{pgfplots} % create normal/logarithmic plots in two and three dimensions
\pgfplotsset{compat=1.18} % sorts out the compatability warning
\usepackage{physics} % useful for vector calculus and linear algebra symbols
\usepackage[most]{tcolorbox} % for producing coloured boxes
\tcbuselibrary{theorems} % theorems with tcolorbox
\usepackage{tikz-3dplot} % for producing 3d plots
\usepackage{tikz} % for drawing graphics in LaTeX
\usepackage{tkz-base} % drawing with a Cartesian coordinate system
\usepackage{tkz-euclide} % drawing in Euclidean geometry
\usepackage{xcolor} % a package for colours


%%% DOCUMENT SETTINGS
\topmargin=-0.45in
\evensidemargin=0in
\oddsidemargin=0in
\textwidth=6.5in
\textheight=9.0in
\headsep=0.25in
\linespread{1.1}
\pagestyle{fancy}
\lhead{\hmwkAuthorName}
\chead{}
% \rhead{\hmwkStudentnum}
\rhead{\MakeLowercase{\leftmark}}
\lfoot{\lastxmark}
\cfoot{\thepage}
\renewcommand\headrulewidth{0.4pt}
\renewcommand\footrulewidth{0.4pt}
\setlength\parindent{0pt}


\renewcommand{\part}[1]{\textbf{\large Part \Alph{partCounter}}\stepcounter{partCounter}\\} % part macro
\newcommand{\solution}{\textbf{\large Solution}} % solution macro

\newmdenv[
    backgroundcolor=gray!20,
    skipabove=\topsep,
    skipbelow=\topsep,
]{grayBoxed}

% General writing
\newcommand{\bracket}[1]{\left(#1\right)} % for automatic resizing of brackets
\newcommand{\sbracket}[1]{\left[#1\right]} % for automatic resizing of square brackets
\newcommand{\mset}[1]{\left\{#1\right\}} % for automatic resizing of curly brackets
\newcommand{\defeq}{\coloneqq} % the "defined as" command
\newcommand{\RNum}[1]{\uppercase\expandafter{\romannumeral #1\relax}} % uppercase roman numerals

\newcommand{\rednote}[1]{{\color{red} #1}} % for red text
\newcommand{\bluenote}[1]{{\color{blue} #1}} % for blue text
\newcommand{\greennote}[1]{{\color{green} #1}} % for green text

% Blackboard Maths Symbols
\newcommand{\N}{\mathbb{N}} % natural numbers
\newcommand{\Q}{\mathbb{Q}} % rational numbers
\newcommand{\Z}{\mathbb{Z}} % integers
\newcommand{\R}{\mathbb{R}} % real numbers
\newcommand{\C}{\mathbb{C}} % complex numbers
\newcommand{\E}{\mathbb{E}} % expectation operators

% Aesthetic
\newcommand{\thmBox}[2]{
    \begin{tcolorbox}[colback=blue!5!white,colframe=blue!50!black,
            colbacktitle=blue!90!black,title=Theorem #1]
        #2
    \end{tcolorbox}
}

\newcommand{\lmBox}[2]{
    \begin{tcolorbox}[colback=blue!5!white,colframe=blue!50!black,
            colbacktitle=blue!50!black,title=Lemma #1]
        #2
    \end{tcolorbox}
}

\newcommand{\defBox}[2]{
    \begin{tcolorbox}[colback=red!5!white,colframe=red!50!black,
            colbacktitle=red!60!black,title=Definition #1]
        #2
    \end{tcolorbox}
}


\newcommand{\exerciseBox}[2]{
    \begin{tcolorbox}[colback=red!5!white,colframe=red!50!black,
            colbacktitle=red!60!black,title=Definition #1]
        #2
    \end{tcolorbox}
}

% Homework Specific
\newcommand{\hBox}[1]{
    \begin{tcolorbox}[colback=blue!5!white,colframe=blue!50!black]
        #1
    \end{tcolorbox}
}

\newcommand{\hmwkStudentnum}{21319203}
\newcommand{\hmwkAuthorName}{\textbf{Angel Cervera Roldan}}

%%% Define the homeworkProblem environment
\newcommand{\enterProblemHeader}[1]{
    \nobreak\extramarks{}{Problem \arabic{#1} continued on next page\ldots}\nobreak{}
    \nobreak\extramarks{Problem \arabic{#1} (continued)}{Problem \arabic{#1} continued on next page\ldots}\nobreak{}
}

\newcommand{\exitProblemHeader}[1]{
    \nobreak\extramarks{Problem \arabic{#1} (continued)}{Problem \arabic{#1} continued on next page\ldots}\nobreak{}
    \stepcounter{#1}
    \nobreak\extramarks{Problem \arabic{#1}}{}\nobreak{}
}

\setcounter{secnumdepth}{0}
\newcounter{partCounter}
\newcounter{homeworkProblemCounter}
\setcounter{homeworkProblemCounter}{1}
\nobreak\extramarks{Problem \arabic{homeworkProblemCounter}}{}\nobreak{}

\newenvironment{homeworkProblem}[1][-1]{
    \ifnum#1>0
        \setcounter{homeworkProblemCounter}{#1}
    \fi
    \section{Problem \arabic{homeworkProblemCounter}}
    \setcounter{partCounter}{1}
    \enterProblemHeader{homeworkProblemCounter}
}{
    \exitProblemHeader{homeworkProblemCounter}
}
 
\newcommand{\classTitle}{Class Name}

\title{
    \vspace{2in}
        \textmd{\textbf{\classTitle}}\\
    \vspace{1in}
    \textmd{\textbf{Homework \#2}}\\
    \vspace{1in}
}

\author{
    \hmwkAuthorName\\
    \hmwkStudentnum\\
}

\date{}

\begin{document}

\maketitle

\pagebreak

% Problem 1
\begin{homeworkProblem}
    \hBox{
        Are the following true, or false

        \begin{enumerate}
            \item There is no subset $A \subseteq \R^2$ such that $Bdy(A)$ contains exactly four points.
            \item There is a subset $C \subseteq \R^3$ such that C is not the empty set and $Int(C)$ is the empty
                  set.
        \end{enumerate}
    }

    \subsection*{Part 1}

    This statement is false. We can prove this by counter example.

    Take $A = \mset{(0, 0), (1, 1), (2, 2), (3, 3)}$. $A$ has only 4 elements, an open ball with a center at any of those elements
    will contain at least one point inside it (the center), as well as points outside it, meaning that they are boundary points.

    \subsection*{Part 2}

    This statement is true, an example would be $C = \mset{(1, 1, 1)}$. This set is not empty, and any open ball with center $(1, 1, 1)$ will contain points outside $C$, meaning that $(1, 1, 1)$ is not an interior point.

    Therefore, there exists $C \subseteq \R^3$ where $Int(C) = \emptyset$

\end{homeworkProblem}
\pagebreak

% Problem 2
\begin{homeworkProblem}
    \hBox{
        Suppose the directional derivative of $g(x, y)$ at $(1, 2)$ in the direction of $\vec{i} + \vec{j}$ is $2\sqrt{2}$
        and the directional derivative of $g(x, y)$ at $(1, 2)$ in the direction of $-2\vec{j}$ is $-3$. Find the
        directional derivative of $g(x, y)$ at $(1, 2)$ in the direction of $-\vec{i} - 2\vec{j}$
    }

    $$
        \gradient g_{\mid (1, 2)} = \pderiv{g}{x} \vec{i} + \pderiv{g}{y} \vec{j}
    $$

    To make the solution clearer, let $a = \pderiv{g}{x}$, and $b = \pderiv{g}{y}$

    First, let's find the unit vectors in the directions given:

    $$
        u = \frac{\vec{i} + \vec{j}}{\sqrt{2}} = \frac{1}{\sqrt{2}} \vec{i} + \frac{1}{\sqrt{2}} \vec{j}
    $$

    $$
        w = \frac{0 \vec{i} - 2 \vec{j}}{\sqrt{(-2)^2}} = 0 \vec{i} - 1 \vec{j}
    $$

    $$
        z = \frac{- 1\vec{i} - 2 \vec{j}}{\sqrt{(-1)^2 + (-2)^2}} = - \frac{1}{\sqrt{5}} \vec{i} - \frac{2}{\sqrt{5}}
    $$

    We know that the gradient in the direction of $w$ is given by:

    $$
        (D_w g)_{\mid (1, 2)} = -3
    $$

    Therefore

    $$
        -3
        = (a \vec{i} + b \vec{j}) \cdot (0\vec{i} - 1 \vec{j})
        = -b
        \implies 3 = b
    $$

    We also know that the gradient in the direction of $u$ is given by:

    $$
        (D_u g)_{\mid (1, 2)} = 2 \sqrt{2}
    $$

    Therefore

    $$
        2 \sqrt{2}
        = (a \vec{i} + b \vec{j}) \cdot (\frac{1}{\sqrt{2}} \vec{i} + \frac{1}{\sqrt{2}} \vec{j})
        = \frac{a}{\sqrt{2}} + \frac{b}{\sqrt{2}}
        = \frac{a}{\sqrt{2}} + \frac{3}{\sqrt{2}}
    $$

    Now, we can solve for $a$

    \begin{align*}
        2 \sqrt{2} & = \frac{a}{\sqrt{2}} + \frac{3}{\sqrt{2}} \\
        2 \cdot 2  & = a + 3                                   \\
        4 - 3      & = a = 1
    \end{align*}

    And so now we now that

    $$
        \gradient g_{\mid (1, 2)} = 1 \vec{i} + 3 \vec{j}
    $$

    With this, we can now find the directional derivative of $g$ at $(1, 2)$ in the direction of $-\vec{i} - 2\vec{j}$


    $$
        (D_z g)_{\mid (1, 2)} = \gradient g_{\mid (1, 2)} \cdot \underline{w} = (\vec{i} + 3 \vec{j}) \cdot (- \frac{1}{\sqrt{5}} \vec{i} - \frac{2}{\sqrt{5}} \vec{j}) = - \frac{1}{\sqrt{5}} - \frac{3 \cdot 2}{\sqrt{5}} = - \frac{7}{\sqrt{5}}
    $$
\end{homeworkProblem}
\pagebreak

% Problem 3
\begin{homeworkProblem}
    \subsection*{Part 1}
    \hBox{
        Suppose that $h(x) = \cos(y + x^2) + \sin(x - 2y^2)$, find the following:

        $$
            \pderiv{h}{x} \cm \pderiv{h}{y} \cm \pderivsq{h}{x} \cm \pderivsq{h}{y} \cm \frac{\partial^2h}{\partial x \partial y}
        $$
    }

    For these differentials, I will use the following rules:

    $$\pderiv{}{u} \cos(f(u)) = - f'(u) \sin(f(u)) \cm \pderiv{}{u} \sin(f(u)) = f'(u) \cos(f(u))$$
    $$\pderiv{}{u} f(u)g(u) = f'(u) g(u) + f(u) g'(u)$$

    And also

    $$
        \pderiv{}{x} y + x^2 = 2x \quad \pderiv{}{x} x - 2y^2 = 1
    $$

    $$
        \pderiv{}{y} y + x^2 = 1 \quad \pderiv{}{y} x - 2y^2 = -4y
    $$

    \begin{align*}
        \pderiv{h}{x}   & = \pderiv{}{x} \cos(y + x^2) + \pderiv{}{x} \sin(x - 2y^2) = - 2x \sin(y + x^2) + \cos(x - 2y^2) \\ \\ \\
        \pderivsq{h}{x} & = \pderiv{}{x} \pderiv{h}{x} =\pderiv{}{x} (- 2x \sin(y + x^2) + \cos(x - 2y^2))                 \\
                        & = -2 \sin(y + x^2) -4x^2 cos(y + x^2) - \sin(x - 2y^2)                                           \\ \\ \\
        \pderiv{h}{y}   & = \pderiv{}{y} \cos(y + x^2) + \pderiv{}{y} \sin(x - 2y^2) = -\sin(y + x^2) - 4y \cos(x - 2y^2)  \\ \\ \\
    \end{align*}

    \begin{align*}
        \pderivsq{h}{y}                           & = \pderiv{}{y} \pderiv{h}{y} =\pderiv{}{y} (-\sin(y + x^2) - 4y \cos(x - 2y^2))  \\
                                                  & = - \cos(y + x^2) - 4 \cos(x - 2y^2) - 16y^2 sin(x - 2y^2)                       \\ \\ \\
        \frac{\partial^2h}{\partial x \partial y} & = \pderiv{}{x} \pderiv{h}{y} = \pderiv{}{x} (-\sin(y + x^2) - 4y \cos(x - 2y^2)) \\
                                                  & =-2x \cos(y + x^2) - 4y  \cos(x -2y^2)
    \end{align*}
    \pagebreak

    \subsection*{Part 2}
    \hBox{
        Suppose that $z = 2u - 3w^2$, $u = e^{2r + 3s}$, $w = s - r^2$ find the following:

        $$
            \pderiv{z}{r} \cm \pderiv{z}{s}
        $$
    }

    For these differentials, I will use the chain rule as follows:

    $$\pderiv{}{a} e^{f(a, b)} = (\pderiv{}{a} f(a, b)) \cdot e^{f(a, b)}$$

    $$\pderiv{}{a} (f(a, b))^2 = 2(f(a, b)) \cdot (\pderiv{}{a} f(a, b))$$


    First, we will find the derivatives of $u$, and $w^2$:


    $$\pderiv{}{r} e^{2r + 3s} = 2e^{2r + 3s}$$
    $$\pderiv{}{s} e^{2r + 3s} = 3e^{2r + 3s}$$
    $$\pderiv{}{r} (s - r^2)^2 = 2(s - r^2) (-2r) = -4r(s - r^2)$$
    $$\pderiv{}{s} (s - r^2)^2 = 2(s - r^2) (1) = 2(s - r^2)$$

    We can use the above derivatives to solve the question

    \begin{align*}
        \pderiv{z}{r} & = 2 \pderiv{u}{r} - 3 \pderiv{w^2}{r} = 2 \pderiv{}{r} e^{2r + 3s} - 3 \pderiv{}{r} (s - r^2)^2  \\
                      & = 2 (2e^{2r + 3s}) - 3 (-4r(s - r^2))                                                            \\
                      & = 4e^{2r + 3s} + 12 r (s - r^2)                                                                  \\ \\
        \pderiv{z}{s} & = 2 \pderiv{u}{s} - 3 \pderiv{w^2}{s}  = 2 \pderiv{}{s} e^{2r + 3s} - 3 \pderiv{}{s} (s - r^2)^2 \\
                      & = 2 (3e^{2r + 3s}) - 3 (2(s - r^2)) = 6e^{2r + 3s} - 6(s - r^2)                                  \\
                      & = 6 (e^{2r + 3s} - s + r^2)
    \end{align*}

\end{homeworkProblem}
\pagebreak


% Problem 4
\begin{homeworkProblem}
    \hBox{
        Suppose $f (x, y) = y^2 - \cos(y + x)$. Is f differentiable at every point in $\R^2$? Justify your answer.
    }

    We say that $f(x, y)$ is differentiable a point $(a, b)$ if $\pderiv{f}{x}_{\mid {a, b}}$ and $\pderiv{f}{y}_{\mid {a, b}}$ exist.

    $$
        \pderiv{f}{x} = \pderiv{}{x} y^2 - \pderiv{}{x} \cos(y + x) = \sin(y + x)
    $$

    Since $x, y \in \R$, $x + y \in \R$, this means that $f$ is differentiable at any point since $\sin(n)$ is defined for all $n \in \R$.

    $$
        \pderiv{f}{y} = \pderiv{}{y} y^2 - \pderiv{}{y} \cos(y + x) = 2y + \sin(y + x)
    $$

    This derivative is also differentiable at any point since $2y$ is defined for all $y \in \R$, and $sin(y + x)$ is also defined for all $x, y \in \R$

    Because both $\pderiv{f}{x}$ and $\pderiv{f}{y}$ are differentiable at every $(x, y) \in \R^2$, we have shown that $f$ is differentiable at every point in $\R^2$

\end{homeworkProblem}
\pagebreak

% Problem 5
\begin{homeworkProblem}
    \subsection*{Part 1}
    \hBox{
        Find the directional derivative of $g(x, y, z) = x^3y^2 - e^{z}\sin(yx)$ at $(1, \frac{\pi}{2} , 0)$ in the
        direction of $\vec{i} - 2\vec{j} + 3\vec{k}$.
    }


    Firstly, we will find the unit vector $u$ in the direction of $\vec{i} - 2\vec{j} + 3\vec{k}$. This is given by

    $$
        u
        = \frac{\vec{i} - 2\vec{j} + 3\vec{k}}{\sqrt{1^2 + (-2)^2 + 3^2}}
        = \frac{1}{\sqrt{14}}\vec{i} - \frac{2}{\sqrt{14}}\vec{j} + \frac{3}{\sqrt{14}}\vec{k}
    $$

    The directional derivative of $g$ in the direction of $\vec{i} - 2\vec{j} + 3\vec{k}$ at $(1, \frac{\pi}{2}, 0)$ is given by

    $$
        (D_{\underline{u}}f)_{\mid (1, \frac{\pi}{2}, 0)} = \gradient g_{\mid (1, \frac{\pi}{2}, 0)} \cdot \underline{u}
    $$

    Where

    $$
        \gradient g = \pderiv{g}{x} \vec{i} + \pderiv{g}{y} \vec{j} + \pderiv{g}{z} \vec{k}
    $$

    \begin{align*}
        \pderiv{g}{x} & = 3x^2y^2 - e^zy\cos(yx) \\ \\
        \pderiv{g}{y} & = 2x^3y - e^zx\cos(yx)   \\ \\
        \pderiv{g}{z} & = - e^z\cos(yx)
    \end{align*}


    \begin{align*}
        (D_{\underline{u}}f)_{\mid (1, \frac{\pi}{2}, 0)}
         & = \gradient g_{\mid (1, \frac{\pi}{2}, 0)} \cdot \underline{u}                                                                                     \\
         & = [(3x^2y^2 - e^zy\cos(yx)) \vec{i} + (2x^3y - e^zx\cos(yx)) \vec{j} + - e^z\cos(yx) \vec{k}]_{\mid (1, \frac{\pi}{2}, 0)} \cdot \underline{u}     \\
         & = (\frac{3 \pi^2}{4} \vec{i}+ \pi \vec{j}- 1 \vec{k}) \cdot (\frac{1}{\sqrt{14}}\vec{i} - \frac{2}{\sqrt{14}}\vec{j} + \frac{3}{\sqrt{14}}\vec{k}) \\
         & = \frac{\sqrt{14} ( 3\pi^2 -8\pi - 12)}{56}
    \end{align*}

    \subsection*{Part 2}
    \hBox{
        Find the directions in which $g(x, y) = 2y^2x - 4e^{yx} \sin x$ increases and decreases most
        rapidly at $(0, 1)$. Also, at what rate does $g$ change in these directions?
    }

    It will most rapidly increase when the angle between the gradient of $g$ and $\underline{u}$ is $0$, and it will most rapidly decrease when it is $\pi$.

    $$
        \gradient g = \pderiv{g}{x} \vec{i} + \pderiv{g}{x} \vec{j}
    $$

    \begin{align*}
        \pderiv{g}{x}
         & = \pderiv{}{x} 2 y^2 x - 4 \pderiv{}{x} e^{yx} \sin{x}                       \\
         & = 2 y^2 - 4 \pderiv{}{x} e^{yx} \sin{x}                                      \\
         & = 2 y^2 - 4 ((\pderiv{}{x} e^{yx}) \sin{x} + e^{yx} (\pderiv{}{x} \sin{x}) ) \\
         & = 2 y^2 - 4 (ye^{yx}  \sin{x} + e^{yx} \cos{x} )                             \\ \\
        \pderiv{g}{y}
         & = \pderiv{}{y} 2 y^2 x - 4 \pderiv{}{y} e^{yx} \sin{x}                       \\
         & = 4yx - 4\sin{x} \pderiv{}{y} e^{yx}                                         \\
         & = 4yx - 4\sin(x) x e^{yx}                                                    \\
    \end{align*}

    Therefore,

    $$
        \gradient g = (2 y^2 - 4 (ye^{yx}  \sin{x} + e^{yx} \cos{x})) \vec{i} + (4yx - 4\sin(x) x e^{yx}) \vec{j}
    $$

    Now, we can evaluate the gradient at $(0, 1)$

    $$
        \gradient g_{\mid (0, 1)} = (2(1)^2 - 4((1)e^0 sin(0) + e^0 cos(0))) \vec{i} + (4(1)(0) - 4 \sin(0) (0) e^{0}) \vec{j} = -2 \vec{i} + 0 \vec{j}
    $$

    The function will increase the most rapidly when $\underline{u}_{inc}$ is in the direction of the gradient evaluated at $(0, 1)$, meaning that $\underline{u}_{inc}$ is in the direction of $-2\vec{i} + 0 \vec{j}$.

    Because $\underline{u}_{inc}$ is a unit vector, $\underline{u}_{inc} = \frac{-2\vec{i} + 0 \vec{j}}{2} = -1 \vec{i} + 0 \vec{j}$

    The change in this direction can be calculated by

    $$
        (D_{u_{inc}}g_{\mid (0, 1)}) = \gradient g_{\mid (0, 1)} \cdot \underline{u}_{inc} = (-2 \vec{i} + 0 \vec{j}) \cdot (-1\vec{i} + 0 \vec{j}) = 2
    $$

    Similarly, it will decrease the most rapid when $\underline{u}_{dec}$ is in the direction of $- \gradient g_{\mid (0, 1)}$. In this case $\underline{u}_{dec} = - \frac{-2\vec{i} + 0 \vec{j}}{2} = 1 \vec{i} + 0 \vec{j}$

    The change in this direction can be calculated by

    $$
        (D_{\underline{u}_{dec}}g_{\mid (0, 1)}) = \gradient g_{\mid (0, 1)} \cdot \underline{u}_{dec} = (-2 \vec{i} + 0 \vec{j}) \cdot (1\vec{i} + 0 \vec{j}) = -2
    $$

\end{homeworkProblem}
\pagebreak




\end{document}
