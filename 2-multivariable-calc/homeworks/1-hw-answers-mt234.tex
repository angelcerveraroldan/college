\documentclass[12pt]{article} % use the article class

\usepackage[plain]{algorithm} % algorithms package
\usepackage{algpseudocode} % pseudo-code package
\usepackage{amsbsy} % for producing bold maths symbols
\usepackage{amsfonts} % an extended set of fonts for maths
\usepackage{amssymb} % various maths symbols
\usepackage{amsthm} % for producing theorem-like environments
\usepackage{datetime2} % managing dates and times
\usepackage{delimseasy} % makes easy the manual sizing of brackets, square brackets, and curly brackets
\usepackage{enumitem} % customing list environments
\usepackage{extramarks} % extra marks
\usepackage{fancyhdr} % headers and footers
\usepackage{float} % makes dealing with floats (e.g. tables and figures) easier
\usepackage{framed} % for producing framed boxes
\usepackage{graphicx} % for including graphics in the document
\usepackage{hyperref} % automatically produce hyperlinks for cross-references
\usepackage{import} % Import and subimport
\usepackage{listings} % Code blocks
\usepackage{mathtools} % package for maths (fixes some deficiences of amsmath so is preferred)
\usepackage{mdframed} % boxes
\usepackage{microtype} % better font sizing (extremely helpful with long equations!)
\usepackage{newtx} % a fonts package
\usepackage{parskip}
\usepackage{pdfpages} % for including pdf documents inside the compiled pdf
\usepackage{pgf} % produce pdf graphics using LaTeX
\usepackage{pgfplots} % create normal/logarithmic plots in two and three dimensions
\pgfplotsset{compat=1.18} % sorts out the compatability warning
\usepackage{physics} % useful for vector calculus and linear algebra symbols
\usepackage[most]{tcolorbox} % for producing coloured boxes
\tcbuselibrary{theorems} % theorems with tcolorbox
\usepackage{tikz-3dplot} % for producing 3d plots
\usepackage{tikz} % for drawing graphics in LaTeX
\usepackage{tkz-base} % drawing with a Cartesian coordinate system
\usepackage{tkz-euclide} % drawing in Euclidean geometry
\usepackage{xcolor} % a package for colours


%%% DOCUMENT SETTINGS
\topmargin=-0.45in
\evensidemargin=0in
\oddsidemargin=0in
\textwidth=6.5in
\textheight=9.0in
\headsep=0.25in
\linespread{1.1}
\pagestyle{fancy}
\lhead{\hmwkAuthorName}
\chead{}
\rhead{\hmwkStudentnum}
\lfoot{\lastxmark}
\cfoot{\thepage}
\renewcommand\headrulewidth{0.4pt}
\renewcommand\footrulewidth{0.4pt}
\setlength\parindent{0pt}


\renewcommand{\part}[1]{\textbf{\large Part \Alph{partCounter}}\stepcounter{partCounter}\\} % part macro
\newcommand{\solution}{\textbf{\large Solution}} % solution macro

\newmdenv[
    backgroundcolor=gray!20,
    skipabove=\topsep,
    skipbelow=\topsep,
]{grayBoxed}

% General writing
\newcommand{\bracket}[1]{\left(#1\right)} % for automatic resizing of brackets
\newcommand{\sbracket}[1]{\left[#1\right]} % for automatic resizing of square brackets
\newcommand{\mset}[1]{\left\{#1\right\}} % for automatic resizing of curly brackets
\newcommand{\defeq}{\coloneqq} % the "defined as" command
\newcommand{\RNum}[1]{\uppercase\expandafter{\romannumeral #1\relax}} % uppercase roman numerals

\newcommand{\rednote}[1]{{\color{red} #1}} % for red text
\newcommand{\bluenote}[1]{{\color{blue} #1}} % for blue text
\newcommand{\greennote}[1]{{\color{green} #1}} % for green text

% Blackboard Maths Symbols
\newcommand{\N}{\mathbb{N}} % natural numbers
\newcommand{\Q}{\mathbb{Q}} % rational numbers
\newcommand{\Z}{\mathbb{Z}} % integers
\newcommand{\R}{\mathbb{R}} % real numbers
\newcommand{\C}{\mathbb{C}} % complex numbers
\newcommand{\E}{\mathbb{E}} % expectation operators

% Aesthetic
\newcommand{\thmBox}[2]{
    \begin{tcolorbox}[colback=blue!5!white,colframe=blue!50!black,
            colbacktitle=blue!90!black,title=Theorem #1]
        #2
    \end{tcolorbox}
}

\newcommand{\lmBox}[2]{
    \begin{tcolorbox}[colback=blue!5!white,colframe=blue!50!black,
            colbacktitle=blue!50!black,title=Lemma #1]
        #2
    \end{tcolorbox}
}

\newcommand{\defBox}[2]{
    \begin{tcolorbox}[colback=red!5!white,colframe=red!50!black,
            colbacktitle=red!60!black,title=Definition #1]
        #2
    \end{tcolorbox}
}

% Homework Specific
\newcommand{\hBox}[1]{
    \begin{tcolorbox}[colback=blue!5!white,colframe=blue!50!black]
        #1
    \end{tcolorbox}
}

\newcommand{\hmwkStudentnum}{21319203}
\newcommand{\hmwkAuthorName}{\textbf{Angel Cervera Roldan}}

%%% Define the homeworkProblem environment
\newcommand{\enterProblemHeader}[1]{
    \nobreak\extramarks{}{Problem \arabic{#1} continued on next page\ldots}\nobreak{}
    \nobreak\extramarks{Problem \arabic{#1} (continued)}{Problem \arabic{#1} continued on next page\ldots}\nobreak{}
}

\newcommand{\exitProblemHeader}[1]{
    \nobreak\extramarks{Problem \arabic{#1} (continued)}{Problem \arabic{#1} continued on next page\ldots}\nobreak{}
    \stepcounter{#1}
    \nobreak\extramarks{Problem \arabic{#1}}{}\nobreak{}
}

\setcounter{secnumdepth}{0}
\newcounter{partCounter}
\newcounter{homeworkProblemCounter}
\setcounter{homeworkProblemCounter}{1}
\nobreak\extramarks{Problem \arabic{homeworkProblemCounter}}{}\nobreak{}

\newenvironment{homeworkProblem}[1][-1]{
    \ifnum#1>0
        \setcounter{homeworkProblemCounter}{#1}
    \fi
    \section{Problem \arabic{homeworkProblemCounter}}
    \setcounter{partCounter}{1}
    \enterProblemHeader{homeworkProblemCounter}
}{
    \exitProblemHeader{homeworkProblemCounter}
}
 
\newcommand{\classTitle}{Multivariable Calculus}

\newcommand{\realxreal}{\mathbb{R}^2}
\newcommand{\pderiv}[2]{\frac{\partial#1}{\partial#2}}
\newcommand{\pderivsq}[2]{\frac{\partial^2#1}{\partial^2#2}}

\newcommand{\cm}{\text{, }}

\title{
    \vspace{2in}
        \textmd{\textbf{\classTitle}}\\
    \vspace{1in}
    \textmd{\textbf{Homework \#1}}\\
    \vspace{1in}
}

\author{
    \hmwkAuthorName\\
    \hmwkStudentnum\\
}

\date{}

\begin{document}

\maketitle

\pagebreak

%% Question 1
\begin{homeworkProblem}
    \hBox{

        $$
            f(x, y) = 3 - \frac{10}{\sqrt{3 - x + 2y}}
        $$

        \begin{enumerate}
            \item Find the domain and the range of $f$
            \item Find the boundary and the interior of the domain
            \item Is the domain open, closed, bounded, or unbounded?
        \end{enumerate}

    }

    Note: For this question, let D = Domain of f, and A = Range of f

    \subsection*{Part 1}

    The domain is the set of all possible inputs. Because there is a fraction, the function will be undefined when the
    denominator is 0.

    Because there is a square root, and our function is in the reals, the function will also be undefined when the inside the
    square root is negative, as that will have no answer in the reals.

    Therefore, the function is undefined when: $3 - x + 2y \leq 0$

    \begin{align*}
        D & = \{(x, y) \in \realxreal \mid 3 - x + 2y > 0\}      \\
          & = \{(x, y) \in \realxreal \mid y > \frac{x - 3}{2}\} \\
    \end{align*}

    The range of $f$ is $(-\infty, 3)$, this is because the fraction can never yield a negative value since both the enumerator, and the
    denominator are strictly greater than 0. Therefore, the range of the fraction is $(0, \infty)$

    \subsection*{Part 2}

    The boundary of the domain of $f$ (which we let be $D$) is every $(x, y)$ such that if $(x, y)$ is the center of any open ball, the the open ball will
    contain points inside and outside $D$

    Let $B = \{(x, y) \in \realxreal \mid 3 - x + 2y = 0\}$

    \textbf{Claim}: $B = Bdy(D)$

    Take $(x_0, y_0) \in B$, and let $r > 0$ be any real number.

    We must show that any open ball with radius $r$ will have points inside and also outside $D$.

    The equation of said open ball can be given by:

    $$
        O = \{(x, y) \in \realxreal | (x - x_0)^2 + (y - y_0)^2 < r^2\}
    $$

    Now we need to show, that there exists an element in $O$ that is also in $D$.

    Take $(a,b) = (x_0 - \frac{r}{2}, y_0 + \frac{r}{2})$, by the definition of $O$, $(a, b) \in O$ iff $(a - x_0)^2 + (b - y_0)^2 < r^2$

    \begin{align*}
        (a - x_0)^2 + (b - y_0)^2
         & = (x_0 - \frac{r}{2} - x_0)^2 + (y_0 + \frac{r}{2} - y_0)^2 < r^2 \\
         & = (-\frac{r}{2})^2 + (\frac{r}{2})^2 < r^2                        \\
         & = \frac{r^2}{2}                                                   \\
         & < r^2                                                             \\
    \end{align*}

    Now that we have shown that $(a, b) \in O$, we need to show that $(a, b) \in D$.
    By the definition of $D$, we know that $(a, b) \in D$ iff $3 - a + 2b > 0$.

    \begin{align*}
        3 - a + 2b
         & = 3 - (x_0 - \frac{r}{2}) + 2(y_0 + \frac{r}{2}) \\
         & =  3 - x_0 + \frac{3r}{2} + 2y_0                 \\
    \end{align*}

    Because we know that $(x_0, y_0) \in D$, we know that

    $$
        3 - x_0 + 2y_0 < 0
    $$

    This means that

    $$
        3 - x_0 + 2y_0 + \frac{3r}{2} < 0 \quad \forall r > 0
    $$

    Meaning that for any $r > 0$, $(a, b) \in D$.

    We have shown that any ball with center $(x_0, y_0) \in B$ will contain at least one element in $D$

    Now we must show that there is an element $(c, d) \in B$ such that $(c, d) \in O$, and $(c, d) \not \in D$.

    To do this, take that same open ball, and let $(c, d) = (x_0 + \frac{r}{2}, y_0 - \frac{r}{2})$.

    $(c, d) \in O$ iff $(c - x_0)^2 + (d - y_0)^2 < r^2$

    \begin{align*}
        (c - x_0)^2 + (d - y_0)^2
         & = (x_0 + \frac{r}{2} - x_0)^2 + (y_0 - \frac{r}{2} - y_0)^2 \\
         & = (\frac{r}{2})^2 + (-\frac{r}{2})^2                        \\
         & = \frac{r^2}{2}                                             \\
         & < r^2                                                       \\
    \end{align*}

    Therefore, $(c, d) \in O$. Now we need to show it is not in $D$.

    $(c, d) \in D$, iff $3 - c + 2d > 0$ is true, however

    \begin{align*}
        3 - c + 2d
         & = 3 -(x_0 + \frac{r}{2}) + 2 (y_0 - \frac{r}{2}) \\
         & = 3 - x_0 - \frac{r}{2} + 2y_0 - r               \\
         & = 3 - x_0 + 2y_0 - \frac{3r}{2}
    \end{align*}

    And because $(x_0, y_0) \in B$, we know that $3 - x_0 + 2y_0 = 0$, therefore

    $$
        3 - c + 2d = 0 - \frac{3r}{2} < 0
    $$

    Since it is smaller than 0, it doesn't meet the requirement to be in $D$, therefore $(c, d) \not \in D$.

    Therefore, any open ball with a center $(x, y) \in B$ will contain points both in $D$ and not in $D$. Therefore, every point
    in $B$ is a boundary point. So we have shown that $B = Bdy(D)$.

    \textbf{Claim}: $Int(D) = D$



    We know that $D \cap Bdy(D) = \emptyset$, as if there was an element $(x, y) \in D \cap Bdy(D)$, then
    $3 - x + 2y > 0$ would be true (because its in $D$), and also $3 - x + 2y = 0$ would be true (because its in $Bdy(D)$),
    but it isn't possible for something to be bigger than 0, but also 0.

    Take an element $(x, y) \in R^2$. If $(x, y) \not \in D$, then any open ball
    with a center $(x, y)$ will not be an element of $Int(D)$, as at least one element in the open ball
    is not in $D$, therefore, if $(x, y) \not \in D \implies (x, y) \not \in Int(D)$

    Let $(x, y) \in D$. Then, $(x, y) \not \in Bdy(D)$, and so $(x, y) \in Int(D)$ (since it isn't in $Bdy(D)$, we know that there exists an open ball around $(x, y)$ such that every element in the open ball is also in $D$).

    We have shown that if $(x, y) \in D$, then $(x, y) \in Int(D)$, and that if $(x, y) \not \in D$, then $(x, y) \not \in Int(D)$, therefore $Int(D) = D$

    \subsection*{Part 3}

    We know that a subset W of the $\realxreal$ is open iff $Int(W) = W$. We showed in the last part that $Int(D) = D$, therefore, $D$ is open.

    To show that $D$ is not closed, we need to find a point in $Bdy(D) \not \in D$.

    Take the point $(5, 1)$, this point is an element of $Bdy(D)$ since $3 - 5 + 2(1) = 0$, it however is not an element of
    $D$, as $3 - 5 + 2(1) \not > 0$. Because of this, $Bdy(D) \not \subset D$, and $D$ is therefore not closed.

    $D$ is unbounded as it is not the subset of any open ball. This can be proven by contradiction.

    Assume that there exist an open ball $P$ with radius $r > 0$ such that $D \subset P$.

    Let the center of the ball be given by $(p_x, p_y)$, then, because it is a ball, the highest point is at $(p_x, p_y + r)$.

    Now, if we find an element in $D$ with a $y$ coordinate larger than $p_y + r$, we know that that element cannot be in $P$.

    Take the element $(2(p_y + 2r), p_y + 2r)$, this element will be in $D$ since

    $$
        3 - 2(p_y + 2r) + 2(p_y + 2r) = 3 > 0
    $$

    We assumed that $D \subset P$, but then we found an element inside $D$ that is not in $P$, this is a contradiction, indicating that $D \not \subset P$.

    Because of this, we know that $D$ is unbounded.


\end{homeworkProblem}

\pagebreak

%Question 2
\begin{homeworkProblem}

    \hBox{
        Find the following limit:

        $$
            \lim_{(x, y) \to (0, -1)} \frac{x^3 - y^2}{2x^5 + 2y^3 - 5}
        $$
    }

    \begin{align*}
        \lim_{(x, y) \to (0, -1)} \frac{x^3 - y^2}{2x^5 + 2y^3 - 5}
         & =  \frac{\lim_{(x, y) \to (0, -1)}x^3 - y^2}{\lim_{(x, y) \to (0, -1)} 2x^5 + 2y^3 - 5} \\
         & = \frac{-1}{-7}                                                                         \\
         & = \frac{1}{7}
    \end{align*}

\end{homeworkProblem}

\pagebreak

%Question 3
\begin{homeworkProblem}

    Are the following statements true ?

    \subsection*{Part 1}
    \hBox{
        If $w$ satisfies $0 \leq 4w < \frac{r^3}{4}$ $\forall r > 0$, then $w = 0$
    }

    Yes. We can prove this by contradiction, assume that $w$ is a real number greater but not equal to 0 ($w > 0$),
    and let $r = w$.

    \begin{align*}
        0 \leq 4w  & < \frac{r^3}{4} = \frac{w^3}{4} \\
        0 \leq 16w & < w^3
    \end{align*}

    The above implies that $4 < w$. However, if $w > 4$, then the initial equation is not satisfied for all $r > 0$,
    for example, for $r = 1$

    $$
        0 \leq 4w < \frac{1^3}{4}
    $$

    The above is not true for any value of w greater than 4, and so this is a contradiction. The assumption we made was that
    there exists $w > 0$, therefore $w$ can only be 0.

    \subsection*{Part 2}
    \hBox{
        For all subsets $A, B$ in $\realxreal$

        $$
            Bdy(A \cup B) = Bdy(A) \cup Bdy(B)
        $$
    }

    To show that it is not true for all subsets $A, B$ of $\realxreal$, we need to find a counter example.

    Take $A$ to be defined as follows:

    $$
        A := \{(0, 0)\}
    $$

    And $B$ to be $\realxreal - \{(0, 0)\}$.

    The boundary points of $A$ are $\{(0, 0)\}$, and the boundary points of $B$ are $\{(0, 0)\}$,
    however, $A \cup B = \realxreal$, which has no boundary points. Therefore $Bdy(A \cup B) \not= Bdy(A) \cup Bdy(B)$




\end{homeworkProblem}

\pagebreak

%% Question 4
\begin{homeworkProblem}

    \hBox{
        Does the following limit exist ?

        $$
            \lim_{(x, y) \to (0, 0)} \frac{y^4 - x^8}{x^8 + y^4}
        $$
    }

    The limit does not exist, we can show this by finding the limit from different paths.

    \subsection*{Path 1}
    Let $(x, y)$ approach from the $y-axis$, meaning that we need to find the limit when $x = 0$, and $y$ approaches (but is not) 0.


    $$
        \lim_{(x, y) \to (0, 0)} \frac{y^4 - 0^8}{0^8 + y^4} = 1
    $$

    \subsection*{Path 2}
    Let $(x, y)$ approach from the $x-axis$, meaning that we need to find the limit when $y = 0$, and $x$ approaches (but is not) 0.

    $$
        \lim_{(x, y) \to (0, 0)} \frac{0^4 - x^8}{x^8 + 0^4} = -1
    $$

    Therefore, since the limit will approach different values depending on which path is taken, we know that the limit does not exist.


\end{homeworkProblem}

\pagebreak

% Question 5
\begin{homeworkProblem}
    \hBox{
        Prove the following:

        $$
            \lim_{(x, y) \to (0, 0)} \frac{3 y^2 x^2}{5 y^2 + x^4} = 0
        $$
    }

    To prove the above, we will use an epsilon delta poof. Let $\epsilon > 0$, we need to show that $\exists \delta > 0$ such that:

    \begin{equation*}
        0 < \sqrt{x^2 + y^2} < \delta \implies \frac{3 y^2 x^2}{5 y^2 + x^4} < \epsilon
    \end{equation*}

    Take $\delta = \sqrt{\epsilon}$, then:

    \begin{align*}
        0 < \sqrt{x^2 + y^2}                    & < \delta            \\
        0 < \sqrt{x^2 + y^2}                    & < \sqrt{\epsilon}   \\
        0 < x^2 + y^2                           & < \epsilon          \\
        0 < x^2 \leq x^2 + y^2                  & < \epsilon          \\
        0 < x^2 \cdot \frac{3 y^2}{5 y^2 + x^4} & \leq x^2 < \epsilon \\
        0 < \frac{3 y^2 x^2}{5 y^2 + x^4}       & < \epsilon          \\
    \end{align*}

    Note: We know the 4th line in the equations above because $y^2, x^4 \geq 0 \quad \forall x, y \in \R$, $3y^2 \leq 5y^2 + x^4$

    $$
        \therefore \frac{3y^2}{5y^2 + x^4} \leq 1 \implies x^2 \cdot \frac{3 y^2}{5 y^2 + x^4} \leq x^2
    $$

    And so we have shown that for any epsilon, there exist a positive value for delta such that:
    \begin{equation*}
        0 < \sqrt{x^2 + y^2} < \delta \implies \frac{3 y^2 x^2}{5 y^2 + x^4} < \epsilon
    \end{equation*}

    Meaning that:

    $$
        \lim_{(x, y) \to (0, 0)} \frac{3 y^2 x^2}{5 y^2 + x^4} = 0
    $$
\end{homeworkProblem}


\end{document}
