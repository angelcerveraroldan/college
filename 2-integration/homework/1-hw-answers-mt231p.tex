\documentclass[12pt]{article} % use the article class

\usepackage[plain]{algorithm} % algorithms package
\usepackage{algpseudocode} % pseudo-code package
\usepackage{amsbsy} % for producing bold maths symbols
\usepackage{amsfonts} % an extended set of fonts for maths
\usepackage{amssymb} % various maths symbols
\usepackage{amsthm} % for producing theorem-like environments
\usepackage{datetime2} % managing dates and times
\usepackage{delimseasy} % makes easy the manual sizing of brackets, square brackets, and curly brackets
\usepackage{enumitem} % customing list environments
\usepackage{extramarks} % extra marks
\usepackage{fancyhdr} % headers and footers
\usepackage{float} % makes dealing with floats (e.g. tables and figures) easier
\usepackage{framed} % for producing framed boxes
\usepackage{graphicx} % for including graphics in the document
\usepackage{hyperref} % automatically produce hyperlinks for cross-references
\usepackage{import} % Import and subimport
\usepackage{listings} % Code blocks
\usepackage{mathtools} % package for maths (fixes some deficiences of amsmath so is preferred)
\usepackage{mdframed} % boxes
\usepackage{microtype} % better font sizing (extremely helpful with long equations!)
\usepackage{newtx} % a fonts package
\usepackage{parskip}
\usepackage{pdfpages} % for including pdf documents inside the compiled pdf
\usepackage{pgf} % produce pdf graphics using LaTeX
\usepackage{pgfplots} % create normal/logarithmic plots in two and three dimensions
\pgfplotsset{compat=1.18} % sorts out the compatability warning
\usepackage{physics} % useful for vector calculus and linear algebra symbols
\usepackage[most]{tcolorbox} % for producing coloured boxes
\tcbuselibrary{theorems} % theorems with tcolorbox
\usepackage{tikz-3dplot} % for producing 3d plots
\usepackage{tikz} % for drawing graphics in LaTeX
\usepackage{tkz-base} % drawing with a Cartesian coordinate system
\usepackage{tkz-euclide} % drawing in Euclidean geometry
\usepackage{xcolor} % a package for colours


%%% DOCUMENT SETTINGS
\topmargin=-0.45in
\evensidemargin=0in
\oddsidemargin=0in
\textwidth=6.5in
\textheight=9.0in
\headsep=0.25in
\linespread{1.1}
\pagestyle{fancy}
\lhead{\hmwkAuthorName}
\chead{}
% \rhead{\hmwkStudentnum}
\rhead{\MakeLowercase{\leftmark}}
\lfoot{\lastxmark}
\cfoot{\thepage}
\renewcommand\headrulewidth{0.4pt}
\renewcommand\footrulewidth{0.4pt}
\setlength\parindent{0pt}


\renewcommand{\part}[1]{\textbf{\large Part \Alph{partCounter}}\stepcounter{partCounter}\\} % part macro
\newcommand{\solution}{\textbf{\large Solution}} % solution macro

\newmdenv[
    backgroundcolor=gray!20,
    skipabove=\topsep,
    skipbelow=\topsep,
]{grayBoxed}

% General writing
\newcommand{\bracket}[1]{\left(#1\right)} % for automatic resizing of brackets
\newcommand{\sbracket}[1]{\left[#1\right]} % for automatic resizing of square brackets
\newcommand{\mset}[1]{\left\{#1\right\}} % for automatic resizing of curly brackets
\newcommand{\defeq}{\coloneqq} % the "defined as" command
\newcommand{\RNum}[1]{\uppercase\expandafter{\romannumeral #1\relax}} % uppercase roman numerals

\newcommand{\rednote}[1]{{\color{red} #1}} % for red text
\newcommand{\bluenote}[1]{{\color{blue} #1}} % for blue text
\newcommand{\greennote}[1]{{\color{green} #1}} % for green text

% Blackboard Maths Symbols
\newcommand{\N}{\mathbb{N}} % natural numbers
\newcommand{\Q}{\mathbb{Q}} % rational numbers
\newcommand{\Z}{\mathbb{Z}} % integers
\newcommand{\R}{\mathbb{R}} % real numbers
\newcommand{\C}{\mathbb{C}} % complex numbers
\newcommand{\E}{\mathbb{E}} % expectation operators

% Aesthetic
\newcommand{\thmBox}[2]{
    \begin{tcolorbox}[colback=blue!5!white,colframe=blue!50!black,
            colbacktitle=blue!90!black,title=Theorem #1]
        #2
    \end{tcolorbox}
}

\newcommand{\lmBox}[2]{
    \begin{tcolorbox}[colback=blue!5!white,colframe=blue!50!black,
            colbacktitle=blue!50!black,title=Lemma #1]
        #2
    \end{tcolorbox}
}

\newcommand{\defBox}[2]{
    \begin{tcolorbox}[colback=red!5!white,colframe=red!50!black,
            colbacktitle=red!60!black,title=Definition #1]
        #2
    \end{tcolorbox}
}


\newcommand{\exerciseBox}[2]{
    \begin{tcolorbox}[colback=red!5!white,colframe=red!50!black,
            colbacktitle=red!60!black,title=Definition #1]
        #2
    \end{tcolorbox}
}

% Homework Specific
\newcommand{\hBox}[1]{
    \begin{tcolorbox}[colback=blue!5!white,colframe=blue!50!black]
        #1
    \end{tcolorbox}
}

\newcommand{\hmwkStudentnum}{21319203}
\newcommand{\hmwkAuthorName}{\textbf{Angel Cervera Roldan}}

%%% Define the homeworkProblem environment
\newcommand{\enterProblemHeader}[1]{
    \nobreak\extramarks{}{Problem \arabic{#1} continued on next page\ldots}\nobreak{}
    \nobreak\extramarks{Problem \arabic{#1} (continued)}{Problem \arabic{#1} continued on next page\ldots}\nobreak{}
}

\newcommand{\exitProblemHeader}[1]{
    \nobreak\extramarks{Problem \arabic{#1} (continued)}{Problem \arabic{#1} continued on next page\ldots}\nobreak{}
    \stepcounter{#1}
    \nobreak\extramarks{Problem \arabic{#1}}{}\nobreak{}
}

\setcounter{secnumdepth}{0}
\newcounter{partCounter}
\newcounter{homeworkProblemCounter}
\setcounter{homeworkProblemCounter}{1}
\nobreak\extramarks{Problem \arabic{homeworkProblemCounter}}{}\nobreak{}

\newenvironment{homeworkProblem}[1][-1]{
    \ifnum#1>0
        \setcounter{homeworkProblemCounter}{#1}
    \fi
    \section{Problem \arabic{homeworkProblemCounter}}
    \setcounter{partCounter}{1}
    \enterProblemHeader{homeworkProblemCounter}
}{
    \exitProblemHeader{homeworkProblemCounter}
}
 
\newcommand{\classTitle}{Class Name}

\newcommand*{\halfpi}{\frac{\pi}{2}}

\title{
    \vspace{2in}
        \textmd{\textbf{\classTitle}}\\
    \vspace{1in}
    \textmd{\textbf{Homework \#1}}\\
    \vspace{1in}
}

\author{
    \hmwkAuthorName\\
    \hmwkStudentnum\\
}

\date{}

\begin{document}

\maketitle

\pagebreak

% Problem 1
\begin{homeworkProblem}
    \begin{align*}
        \cos{(x - \halfpi)}
         & = \cos(x)\cos(-\halfpi) - \sin(x)\sin(-\halfpi) \\
         & = \cos(x) \cdot 0 - \sin(x) \cdot (-1)          \\
         & = \sin(x)
    \end{align*}

    \begin{align*}
        \cos{(x + \frac{\pi}{2})}
         & = \cos(x)\cos(\halfpi) - \sin(x)\sin(\halfpi) \\
         & = \cos(x) \cdot 0 - \sin(x) \cdot (1)         \\
         & = - \sin(x)
    \end{align*}

    \begin{align*}
        \sin{(x - \halfpi)}
         & = \sin(x)cos(-\halfpi) + \cos(x)\sin(-\halfpi) \\
         & = \sin(x) \cdot 0 + \cos(x) \cdot (-1)         \\
         & = -\cos(x)
    \end{align*}

    \begin{align*}
        \sin{(x + \halfpi)}
         & = \sin(x)cos(\halfpi) + \cos(x)\sin(\halfpi) \\
         & = \sin(x) \cdot 0 + \cos(x) \cdot (1)        \\
         & = \cos(x)
    \end{align*}

    \begin{align*}
        \cos{(x - \pi)}
         & = \cos(x)\cos(-\pi) - \sin(x)\sin(-\pi) \\
         & = \cos(x) \cdot (-1) - \sin(x) \cdot 0  \\
         & = -\cos(x)
    \end{align*}

    \begin{align*}
        \cos{(x + \pi)}
         & = \cos(x)\cos(\pi) - \sin(x)\sin(\pi)  \\
         & = \cos(x) \cdot (-1) - \sin(x) \cdot 0 \\
         & = -\cos(x)
    \end{align*}

    \begin{align*}
        \sin{(x - \pi)}
         & = \sin(x)\cos(-\pi) + \cos(x)\sin(-\pi) \\
         & = \sin(x) \cdot (-1) + \cos(x) \cdot 0  \\
         & = -\sin(x)
    \end{align*}

    \begin{align*}
        \sin{(x + \pi)}
         & = \sin(x)\cos(\pi) + \cos(x)\sin(\pi)  \\
         & = \sin(x) \cdot (-1) + \cos(x) \cdot 0 \\
         & = -\sin(x)
    \end{align*}

    \begin{align*}
        \cos{(x - \frac{3\pi}{2})}
         & = \cos(x - \pi - \halfpi)                                         \\
         & = \cos(x - \pi) \cos(-\halfpi) - \sin(x - \pi) \sin(-\halfpi) & * \\
         & = - \cos(x) \cdot 0 + \sin(x) \cdot (-1)                          \\
         & = -\sin(x)                                                        \\
    \end{align*}

    \begin{align*}
        \cos{(x + \frac{3\pi}{2})}
         & = \cos(x + \pi + \halfpi)                                       \\
         & = \cos(x + \pi) \cos(\halfpi) - \sin(x + \pi) \sin(\halfpi) & * \\
         & = - \cos(x) \cdot 0 + \sin(x) \cdot (1)                         \\
         & = \sin(x)                                                       \\
    \end{align*}

    \begin{align*}
        \sin{(x - \frac{3\pi}{2})}
         & = \sin(x - \pi - \halfpi)                                         \\
         & = \sin(x - \pi) \cos(-\halfpi) + \cos(x - \pi) \sin(-\halfpi) & * \\
         & = - \sin(x) \cdot 0 - \cos(x) \cdot (-1)                          \\
         & = \cos(x)
    \end{align*}

    \begin{align*}
        \sin{(x + \frac{3\pi}{2})}
         & = \sin(x + \pi + \halfpi)                                       \\
         & = \sin(x + \pi) \cos(\halfpi) + \cos(x + \pi) \sin(\halfpi) & * \\
         & = - \sin(x) \cdot 0 - \cos(x) \cdot (1)                         \\
         & = - \cos(x)
    \end{align*}

    *
    We had found the values of $\cos(x - \pi)$,  $\cos(x + \pi)$, $\sin(x - \pi)$, and $\sin(x + \pi)$ earlier in this question,
    so I replaced it for what we found earlier.

\end{homeworkProblem}

\pagebreak

% Problem 2
\begin{homeworkProblem}

    \hBox{
        Find the inverse of the following function:

        $$
            f(x)= x^2 - 2bx
        $$

        Where $b$ is a positive constant such that $b \geq x$.
    }

    To find the inverse, we will start by manipulating the original function until
    there is only one x. To do this we will try to find the value of $a, c$ such that $f(x) = (x + a)^2 + c$

    \begin{align*}
        f(x) & = x^2 - 2bx           \\
             & = (x + a)^2 + c       \\
             & = x^2 + 2xa + a^2 + c \\
    \end{align*}

    Therefore:

    \begin{align*}
        2xa           & = -2bx   \\
        \Rightarrow a & = -b     \\
        a^2 + c       & = 0      \\
        \Rightarrow c & = -(b)^2
    \end{align*}

    Now that we have found the values for $a, c$, lets verify that they actually work.

    \begin{align*}
        f(x) & = (x - b)^2 - b^2       \\
             & = x^2 - 2bx + b^2 - b^2 \\
             & = x^2 - 2bx             \\
    \end{align*}

    And so those values work. Now we can find its inverse easier:

    \begin{align*}
        f(x)                  & = (x - b)^2 - b^2     \\
        f(x) + b^2            & = (x - b)^2           \\
        \sqrt{f(x) + b^2}     & = \pm (x - b)         \\
        \sqrt{f(x) + b^2}     & = - (x - b)       & * \\
        \sqrt{f(x) + b^2} - b & = - x                 \\
        b - \sqrt{f(x) + b^2} & = x                   \\
    \end{align*}

    Note that we only keep the negative sign as $b \geq x$, and so $(x - b)$ will always be negative.
    However, it must equal $\sqrt{f(x) + b^2}$, which cannot be negative number.

    \subsection*{Answer}
    $$
        f^{-1}(x) = b - \sqrt{x + b^2}
    $$

    To verify that that's actually the inverse, we need to show that $f(f^{-1}(x)) = f^{-1}(f(x)) = x$

    \begin{align*}
        f(f^{-1}(x))
         & = f(b - \sqrt{x + b^2})                                      \\
         & = (b - \sqrt{x + b^2})^2 - 2b(b - \sqrt{x + b^2})            \\
         & = b^2 - 2b\sqrt{x + b^2} + x + b^2 - 2b^2 + 2b\sqrt{x + b^2} \\
         & = x - 2b\sqrt{x + b^2} + 2b\sqrt{x + b^2} +  2b^2 - 2b^2     \\
         & = x
    \end{align*}

    \begin{align*}
        f^{-1}(f(x))
         & = b - \sqrt{f(x) + b^2}      \\
         & = b - \sqrt{x^2 - 2bx + b^2} \\
         & = b - \sqrt{(x - b)^2}       \\
         & = b - x - b                  \\
         & = x
    \end{align*}

    And because both $f(f^{-1}(x))$ and $f^{-1}(f(x))$ are equal to $x$, $f(f^{-1}(x)) = f^{-1}(f(x))$

\end{homeworkProblem}

\pagebreak

% Problem 3
\begin{homeworkProblem}

    \hBox{
        Find:

        $$
            \lim_{x \to 0} \frac{1 - \cos{x}}{x\sin{x}}
        $$
    }

    \begin{align*}
        \frac{1 - \cos{x}}{x\sin{x}}
         & = \frac{1 - \cos^2{x}}{x\sin{x}(1 + \cos{x})}                 \\
         & = \frac{sin^2{x}}{x\sin{x}(1 + \cos{x})}                      \\
         & = \frac{sin{x}}{x(1 + \cos{x})}                               \\
         & = \frac{sin{x}}{x} \cdot \frac{1}{1 + \cos{x}}                \\
        \therefore
        \lim_{x \to 0} \frac{1 - \cos{x}}{x\sin{x}}
         & = \lim_{x \to 0} \frac{sin{x}}{x} \cdot \frac{1}{1 + \cos{x}} \\
    \end{align*}

    We know from class that:
    \begin{align*}
        \lim_{x \to 0} \frac{sin{x}}{x}      & = 1                     \\
        \lim_{x \to 0} \frac{1}{1 + \cos{x}} & = \frac{1}{1 + \cos{0}} \\
                                             & = \frac{1}{2}
    \end{align*}

    Therefore,

    \begin{align*}
        \lim_{x \to 0} \frac{sin{x}}{x} \cdot \frac{1}{1 + \cos{x}}
         & = \lim_{x \to 0} \frac{sin{x}}{x} \cdot \lim_{x \to 0} \frac{1}{1 + \cos{x}} \\
         & = 1 \cdot \frac{1}{2}                                                        \\
         & = \frac{1}{2}                                                                \\
    \end{align*}

    And so the answer is $\frac{1}{2}$.
\end{homeworkProblem}

\pagebreak

% Problem 4
\begin{homeworkProblem}
    \hBox{
        Prove the following using $\epsilon-\delta$
        $$
            \lim_{x \to 3} \sqrt{4 - x} = 1
        $$
    }

    To show that the limit is true, we need to show that $\forall\epsilon > 0$, $\exists\delta > 0$ such that

    $$
        0 < \abs*{x - 3} < \delta \implies  \abs{\sqrt{4 - x} - 1} < \epsilon
    $$


    \begin{align*}
                                 & \abs*{\sqrt{4- x} - 1}< \epsilon             \\
        - \epsilon             < & \sqrt{4 - x} - 1      < \epsilon             \\
        - \epsilon + 1         < & \sqrt{4 - x}          < \epsilon + 1         \\
        (- \epsilon + 1)^2     < & 4 - x                 < (\epsilon + 1)^2     \\
        (- \epsilon + 1)^2 - 1 < & 3 - x                 < (\epsilon + 1)^2 - 1 \\
    \end{align*}

    Therefore:

    $$
        \abs*{3 - x} = \abs*{x - 3} < \max(\abs*{(- \epsilon + 1)^2 - 1}, \abs*{(\epsilon + 1)^2 - 1})
    $$

    And so we can let $\delta = \max(\abs*{(- \epsilon + 1)^2 - 1}, \abs*{(\epsilon + 1)^2 - 1})$. Therefore, we
    have found a value for delta such that the limit definition is satisfied, meaning that $\lim_{x \to 3} \sqrt{4 - x} = 1$.
\end{homeworkProblem}

\pagebreak

\begin{homeworkProblem}
    \hBox{
        Find the asymptotes for the following function:

        $$
            f(x) = \frac{x^2 - 4}{x^2 - 4x + 3}
        $$
    }

    The given function will have vertical asymptotes when the denominator of the function is 0, as the function will tend towards either positive or negative infinity at these points.
    $f(x)$ can be re-written as:

    $$
        f(x) = \frac{x^2 - 4}{(x - 1) \cdot (x - 3)}
    $$

    This shows that $f$ has vertical asymptotes at $x = 3$, and at $x = 1$.

    To see if there are any horizontal asymptotes we need to find what the function approaches as x goes to infinity.

    $$
        \lim_{x \to \infty} \frac{x^2 - 4}{x^2 - 4x + 3} = \lim_{x \to \infty} \frac{(x - 2) \cdot (x + 2)}{(x - 3) \cdot (x - 1)} = (\lim_{x \to \infty} \frac{x - 2}{x - 3}) \cdot (\lim_{x \to \infty} \frac{x + 2}{x - 1})= 1 \cdot 1 = 1
    $$

    Because it approaches 1, but it never reaches one, we know that there is a horizontal asymptote at $y = 1$.

\end{homeworkProblem}

\pagebreak

\begin{homeworkProblem}
    \hBox{
        At which point/s does the following function fail to be continuous?

        $$
            f(x) =
            \begin{cases}
                x+1                   & \text{for } x \geq 0 \\
                \frac{x}{x^2 - x - 6} & \text{for } x < 0
            \end{cases}
        $$
    }

    $x + 1$ is continuous for all x, therefore, $f(x)$ is continuous $\forall x > 0$.

    $f(x)$ will be continuous at $x = 0$ if the limit as x approaches 0 from both sides is the same and if it is defined at $x = 0$.

    $f(0) = 1$, however, the limit from the negative side is the following:

    \begin{align*}
        \lim_{x \to 0^-} f(x)
         & = \lim_{x \to 0} \frac{x}{x^2 - x - 6} \\
         & = 0
    \end{align*}

    Therefore, $f(x)$ is not continuous at $x = 0$. It will also not be continuous at any points where it is undefined.

    For any $x < 0$, $f(x)$ will be undefined iff $x^2 - x - 6 = 0$, using the $-b$ formula, we get that $ \frac{x}{x^2 - x - 6}$ is undefined at 3, and at -2,
    however, $3 > 0$, so it is defined at that point.

    Therefore, the function fails to be continuous only at $x = -2$ and at $x = 0$
\end{homeworkProblem}

\end{document}
