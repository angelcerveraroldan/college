\documentclass[12pt]{article} % use the article class

\usepackage[plain]{algorithm} % algorithms package
\usepackage{algpseudocode} % pseudo-code package
\usepackage{amsbsy} % for producing bold maths symbols
\usepackage{amsfonts} % an extended set of fonts for maths
\usepackage{amssymb} % various maths symbols
\usepackage{amsthm} % for producing theorem-like environments
\usepackage{datetime2} % managing dates and times
\usepackage{delimseasy} % makes easy the manual sizing of brackets, square brackets, and curly brackets
\usepackage{enumitem} % customing list environments
\usepackage{extramarks} % extra marks
\usepackage{fancyhdr} % headers and footers
\usepackage{float} % makes dealing with floats (e.g. tables and figures) easier
\usepackage{framed} % for producing framed boxes
\usepackage{graphicx} % for including graphics in the document
\usepackage{hyperref} % automatically produce hyperlinks for cross-references
\usepackage{import} % Import and subimport
\usepackage{listings} % Code blocks
\usepackage{mathtools} % package for maths (fixes some deficiences of amsmath so is preferred)
\usepackage{mdframed} % boxes
\usepackage{microtype} % better font sizing (extremely helpful with long equations!)
\usepackage{newtx} % a fonts package
\usepackage{parskip}
\usepackage{pdfpages} % for including pdf documents inside the compiled pdf
\usepackage{pgf} % produce pdf graphics using LaTeX
\usepackage{pgfplots} % create normal/logarithmic plots in two and three dimensions
\pgfplotsset{compat=1.18} % sorts out the compatability warning
\usepackage{physics} % useful for vector calculus and linear algebra symbols
\usepackage[most]{tcolorbox} % for producing coloured boxes
\tcbuselibrary{theorems} % theorems with tcolorbox
\usepackage{tikz-3dplot} % for producing 3d plots
\usepackage{tikz} % for drawing graphics in LaTeX
\usepackage{tkz-base} % drawing with a Cartesian coordinate system
\usepackage{tkz-euclide} % drawing in Euclidean geometry
\usepackage{xcolor} % a package for colours


%%% DOCUMENT SETTINGS
\topmargin=-0.45in
\evensidemargin=0in
\oddsidemargin=0in
\textwidth=6.5in
\textheight=9.0in
\headsep=0.25in
\linespread{1.1}
\pagestyle{fancy}
\lhead{\hmwkAuthorName}
\chead{}
\rhead{\hmwkStudentnum}
\lfoot{\lastxmark}
\cfoot{\thepage}
\renewcommand\headrulewidth{0.4pt}
\renewcommand\footrulewidth{0.4pt}
\setlength\parindent{0pt}


\renewcommand{\part}[1]{\textbf{\large Part \Alph{partCounter}}\stepcounter{partCounter}\\} % part macro
\newcommand{\solution}{\textbf{\large Solution}} % solution macro

\newmdenv[
    backgroundcolor=gray!20,
    skipabove=\topsep,
    skipbelow=\topsep,
]{grayBoxed}

% General writing
\newcommand{\bracket}[1]{\left(#1\right)} % for automatic resizing of brackets
\newcommand{\sbracket}[1]{\left[#1\right]} % for automatic resizing of square brackets
\newcommand{\mset}[1]{\left\{#1\right\}} % for automatic resizing of curly brackets
\newcommand{\defeq}{\coloneqq} % the "defined as" command
\newcommand{\RNum}[1]{\uppercase\expandafter{\romannumeral #1\relax}} % uppercase roman numerals

\newcommand{\rednote}[1]{{\color{red} #1}} % for red text
\newcommand{\bluenote}[1]{{\color{blue} #1}} % for blue text
\newcommand{\greennote}[1]{{\color{green} #1}} % for green text

% Blackboard Maths Symbols
\newcommand{\N}{\mathbb{N}} % natural numbers
\newcommand{\Q}{\mathbb{Q}} % rational numbers
\newcommand{\Z}{\mathbb{Z}} % integers
\newcommand{\R}{\mathbb{R}} % real numbers
\newcommand{\C}{\mathbb{C}} % complex numbers
\newcommand{\E}{\mathbb{E}} % expectation operators

% Aesthetic
\newcommand{\thmBox}[2]{
    \begin{tcolorbox}[colback=blue!5!white,colframe=blue!50!black,
            colbacktitle=blue!90!black,title=Theorem #1]
        #2
    \end{tcolorbox}
}

\newcommand{\lmBox}[2]{
    \begin{tcolorbox}[colback=blue!5!white,colframe=blue!50!black,
            colbacktitle=blue!50!black,title=Lemma #1]
        #2
    \end{tcolorbox}
}

\newcommand{\defBox}[2]{
    \begin{tcolorbox}[colback=red!5!white,colframe=red!50!black,
            colbacktitle=red!60!black,title=Definition #1]
        #2
    \end{tcolorbox}
}

% Homework Specific
\newcommand{\hBox}[1]{
    \begin{tcolorbox}[colback=blue!5!white,colframe=blue!50!black]
        #1
    \end{tcolorbox}
}

\newcommand{\hmwkStudentnum}{21319203}
\newcommand{\hmwkAuthorName}{\textbf{Angel Cervera Roldan}}

%%% Define the homeworkProblem environment
\newcommand{\enterProblemHeader}[1]{
    \nobreak\extramarks{}{Problem \arabic{#1} continued on next page\ldots}\nobreak{}
    \nobreak\extramarks{Problem \arabic{#1} (continued)}{Problem \arabic{#1} continued on next page\ldots}\nobreak{}
}

\newcommand{\exitProblemHeader}[1]{
    \nobreak\extramarks{Problem \arabic{#1} (continued)}{Problem \arabic{#1} continued on next page\ldots}\nobreak{}
    \stepcounter{#1}
    \nobreak\extramarks{Problem \arabic{#1}}{}\nobreak{}
}

\setcounter{secnumdepth}{0}
\newcounter{partCounter}
\newcounter{homeworkProblemCounter}
\setcounter{homeworkProblemCounter}{1}
\nobreak\extramarks{Problem \arabic{homeworkProblemCounter}}{}\nobreak{}

\newenvironment{homeworkProblem}[1][-1]{
    \ifnum#1>0
        \setcounter{homeworkProblemCounter}{#1}
    \fi
    \section{Problem \arabic{homeworkProblemCounter}}
    \setcounter{partCounter}{1}
    \enterProblemHeader{homeworkProblemCounter}
}{
    \exitProblemHeader{homeworkProblemCounter}
}
 
\newcommand{\classTitle}{Multivariable Calculus}

\newcommand{\realxreal}{\mathbb{R}^2}
\newcommand{\pderiv}[2]{\frac{\partial#1}{\partial#2}}
\newcommand{\pderivsq}[2]{\frac{\partial^2#1}{\partial^2#2}}

\newcommand{\cm}{\text{, }}

\newcommand*{\halfpi}{\frac{\pi}{2}}

\title{
    \vspace{2in}
        \textmd{\textbf{\classTitle}}\\
    \vspace{1in}
    \textmd{\textbf{Homework \#2}}\\
    \vspace{1in}
}

\author{
    \hmwkAuthorName\\
    \hmwkStudentnum\\
}

\date{}

\begin{document}

\maketitle

\pagebreak

% Problem 1
\begin{homeworkProblem}
    \hBox{
        Find the tangent line for

        \begin{enumerate}
            \item $f(x) = \frac{8}{\sqrt{x - 2}}$ at $x = 6$
            \item $f(x) = 4 + \cot x - 2 \csc x$ at $x = \halfpi$
        \end{enumerate}
    }

    \subsection*{Part 1}

    First we will find the rate of change of $f$ at $x = 6$

    To solve the derivate, we will let $u = \sqrt{x - 2}$, then:

    $$
        \deriv{f}{x} = \deriv{f}{u} \deriv{u}{x} = (\deriv{}{u} \frac{8}{u}) (\deriv{}{x} \sqrt{x - 2})
    $$

    $$
        \deriv{}{u} \frac{8}{u} = - \frac{8}{u^2} = - \frac{8}{\sqrt{x - 2}^2} = - \frac{8}{x - 2}
    $$

    Now, all we have left to solve is

    $$
        \deriv{}{x} \sqrt{x - 2}
    $$

    To do this, we will let $h = x - 2$, so $u(h) = \sqrt{h}$

    Now, by the chain rule, we have

    $$
        \deriv{u}{x} = \deriv{u}{h} \deriv{h}{x} = \frac{1}{2 \sqrt{h}} \cdot 1 = \frac{1}{2 \sqrt{h}} = \frac{1}{2 \sqrt{x - 2}}
    $$

    So, when putting it all together, we get

    $$
        \deriv{f}{x} = (- \frac{8}{x - 2}) \cdot \frac{1}{2\sqrt{x - 2}} = - \frac{4}{(x - 2) \sqrt{x - 2}}
    $$

    To find the slope of the tangent at $x = 6$, we can evaluate the derivative we just calculated

    $$
        \deriv{f(6)}{x} = - \frac{4}{(6 - 2) \sqrt{6 - 2}} = - \frac{4}{4 \sqrt{4}} = - \frac{1}{2}
    $$

    The tangent line will be in the form:

    $$
        t(x) = mx + c = - \frac{1}{2}x + c
    $$

    To find the value of $c$, we use the fact that $f(6) = t(6)$

    $$
        f(6) = \frac{8}{\sqrt{6 - 2}} = 4 = - \frac{1}{2}(6) + c \implies c = 7
    $$


    Therefore, the line tangent to $f(x)$ at $x = 6$ is given by $t(x) = - \frac{1}{2}x + 7$

    \pagebreak
    \subsection*{Part 2}

    First we will find the rate of change of $f$ at $x = \halfpi$

    $$
        \deriv{f}{x} = \deriv{}{x} (4 + \cot(x) - 2 \csc(x)) = 0 - \csc^2{x} + 2 \csc{x} \cot{x}
    $$

    We can simplify the equation by replacing $\csc{x}$ by $\frac{1}{\sin{x}}$ and $\cot{x}$ by $\frac{\cos{x}}{\sin{x}}$

    $$
        \deriv{f}{x} =  - \frac{1}{\sin^2{x}} + 2 \frac{1}{\sin{x}} \frac{\cos{x}}{\sin{x}} = \frac{-1 + 2\cos{x}}{\sin^2{x}}
    $$

    And if we evaluate the derivative at $x = \halfpi$, we get $\deriv{f}{x}(\halfpi) = -1$, this will be the slope of the tangent line $t(x)$.

    $$
        t(x) = -1x + c
    $$

    We can use the fact that $t(x) = f(x)$ at $x = \halfpi$ to find the value of $c$

    \begin{align*}
        t(\halfpi) & = -\halfpi + c                            \\
        f(\halfpi) & = 4 + \cot{\halfpi} - 2\csc{\halfpi}  = 2 \\
        2          & = -\halfpi + c                            \\
        c          & = 2 + \halfpi
    \end{align*}

    And so the equation of the tangent line of $f$ at $x = \halfpi$ is given by $t(x) = -x + 2 + \halfpi$
\end{homeworkProblem}
\pagebreak


% Problem 2
\begin{homeworkProblem}
    \hBox{
        Find $\frac{dy}{dx}$ for the following functions by stating how you do substitutions:

        \begin{itemize}
            \item $y = e^{2\cos(\pi x - 1)}$
            \item $y = (x^{-\frac{3}{4}}+ x\sin(x))^{\frac{4}{3}}$
        \end{itemize}
    }

    \subsection*{Part 1}

    To find the derivative of $y = e^{2\cos(\pi x - 1)}$, we let $u = \cos(\pi x - 1)$, then

    $$
        \deriv{y}{x} = \deriv{y}{u} \deriv{u}{x}
    $$

    Where

    $$
        \deriv{y}{u} = \deriv{}{u} e^{2u} = 2e^{2u}
    $$

    and

    $$
        \deriv{u}{x} = \deriv{}{x} \cos(\pi x - 1)
    $$

    to solve this derivative, we will once again use substitution. Let $h = \pi x - 1$

    $$
        \deriv{u}{x} = \deriv{u}{h} \deriv{h}{x}
    $$

    $$
        \deriv{h}{x} = \deriv{}{x} \pi x - 1 = \pi
    $$

    $$
        \deriv{u}{h} = \deriv{}{h} \cos(h) = - \sin(h) = - \sin(\pi x - 1)
    $$

    Putting it all together, we get

    $$
        \deriv{y}{x} = \deriv{y}{u} \deriv{u}{x} = \deriv{y}{u} \deriv{u}{h} \deriv{h}{x} = 2e^{2\cos(\pi x - 1)} \cdot (- \sin(\pi x - 1)) \cdot (\pi)
    $$

    Or in a slightly neater way

    $$
        \deriv{y}{x} = -2 \pi e^{2\cos(\pi x - 1)}\sin(\pi x - 1)
    $$

    \pagebreak

    \subsection*{Part 2}

    To find the derivative of $y = (x^{-\frac{3}{4}}+ x\sin(x))^{\frac{4}{3}}$, we let $u = x^{-\frac{3}{4}}+ x\sin(x)$, then

    $$
        \deriv{y}{x} = \deriv{y}{u} \deriv{u}{x}
    $$

    Where

    $$
        \deriv{y}{u} = \deriv{}{u} u^{\frac{4}{3}} = \frac{4}{3} u^{\frac{1}{3}} = \frac{4}{3} (x^{-\frac{3}{4}}+ x\sin(x))^{\frac{1}{3}}
    $$

    and

    \begin{align*}
        \deriv{u}{x}
         & = \deriv{}{x} x^{-\frac{3}{4}} + x\sin(x)             \\
         & = \deriv{}{x} x^{-\frac{3}{4}} +\deriv{}{x}  x\sin(x) \\
         & = - \frac{3}{4} x^{-\frac{7}{4}} + \sin(x) + x\cos(x)
    \end{align*}

    And so when we put it all together, we get

    $$
        \deriv{y}{x} = \deriv{y}{u} \deriv{u}{x} = (\frac{4}{3} (x^{-\frac{3}{4}}+ x\sin(x))^{\frac{1}{3}}) (- \frac{3}{4} x^{-\frac{7}{4}} + \sin(x) + x\cos(x))
    $$
\end{homeworkProblem}
\pagebreak

% Problem 3
\begin{homeworkProblem}
    \hBox{
        Find $\frac{d^2y}{dx^2}$ for $y^2 = e^{x^2} + 2x$.
    }

    \begin{align}
        y^2    & = e^{x^2} + 2x                                                                           \\
        2 y y' & = 2xe^{x^2} + 2                                                                          \\
        y '    & = \frac{2xe^{x^2} + 2}{2y} = \frac{xe^{x^2} + 1}{y}                                      \\
        y ''   & = \deriv{}{x} y ' = \deriv{}{x} \frac{xe^{x^2} + 1}{y}                                   \\
               & = \frac{y \cdot \deriv{}{x} (xe^{x^2} + 1) - (xe^{x^2} + 1) \cdot \deriv{}{x} y}{y ^ 2}  \\
               & = \frac{y \cdot e^{x^2} (2x^2 + 1) - (xe^{x^2} + 1) \cdot y '}{y ^ 2}                    \\
               & = \frac{y \cdot e^{x^2} (2x^2 + 1) - (xe^{x^2} + 1) \cdot \frac{xe^{x^2} + 1}{y}}{y ^ 2} \\
               & = \frac{y^2 \cdot e^{x^2} (2x^2 + 1) - (xe^{x^2} + 1)^2}{y ^ 3}
    \end{align}

    To go from step 5 to step 6, we need the following:

    \begin{align*}
        \deriv{}{x} xe^{x^2} + 1
         & = \deriv{}{x} xe^{x^2}                          \\
         & = x \deriv{}{x} e^{x^2} - e^{x^2} \deriv{}{x} x \\
         & = 2 x^2 e^{x^2} - e^{x^2}                       \\
         & = e^{x^2} (2x^2 - 1)
    \end{align*}

\end{homeworkProblem}
\pagebreak

% Problem 4

\begin{homeworkProblem}
    \hBox{
        Identify the extreme points of the function $f(x) = \frac{x^4}{4} - 2x^2 + 4$, find where the curve
        is increasing and decreasing, and sketch a rough graph for $f(x)$.
    }

    The extreme points of a function are the points where its derivative is equal to 0. Therefore, we must start by finding $\deriv{f}{x}$

    $$
        \deriv{f}{x} = \deriv{}{x} \frac{x^4}{4} - 2 \deriv{}{x} x^2 + \deriv{}{x} 4 = x^3 - 4x
    $$

    $$
        x^3 - 4x = 0 \implies x^3 = 4x
    $$

    This means that $x = 0$ or $x^2 = 2$, therefore the possible values for $x$ are 0, 2, and -2.
\end{homeworkProblem}

\end{document}
