\usepackage[plain]{algorithm} % algorithms package
\usepackage{algpseudocode} % pseudo-code package
\usepackage{amsbsy} % for producing bold maths symbols
\usepackage{amsfonts} % an extended set of fonts for maths
\usepackage{amssymb} % various maths symbols
\usepackage{amsthm} % for producing theorem-like environments
\usepackage{datetime2} % managing dates and times
\usepackage{delimseasy} % makes easy the manual sizing of brackets, square brackets, and curly brackets
\usepackage{enumitem} % customing list environments
\usepackage{extramarks} % extra marks
\usepackage{fancyhdr} % headers and footers
\usepackage{float} % makes dealing with floats (e.g. tables and figures) easier
\usepackage{framed} % for producing framed boxes
\usepackage{graphicx} % for including graphics in the document
\usepackage{hyperref} % automatically produce hyperlinks for cross-references
\usepackage{import} % Import and subimport
\usepackage{listings} % Code blocks
\usepackage{mathtools} % package for maths (fixes some deficiences of amsmath so is preferred)
\usepackage{mdframed} % boxes
\usepackage{microtype} % better font sizing (extremely helpful with long equations!)
\usepackage{newtx} % a fonts package
\usepackage{parskip}
\usepackage{pdfpages} % for including pdf documents inside the compiled pdf
\usepackage{pgf} % produce pdf graphics using LaTeX
\usepackage{pgfplots} % create normal/logarithmic plots in two and three dimensions
\pgfplotsset{compat=1.18} % sorts out the compatability warning
\usepackage{physics} % useful for vector calculus and linear algebra symbols
\usepackage[most]{tcolorbox} % for producing coloured boxes
\tcbuselibrary{theorems} % theorems with tcolorbox
\usepackage{tikz-3dplot} % for producing 3d plots
\usepackage{tikz} % for drawing graphics in LaTeX
\usepackage{tkz-base} % drawing with a Cartesian coordinate system
\usepackage{tkz-euclide} % drawing in Euclidean geometry
\usepackage{xcolor} % a package for colours


%%% DOCUMENT SETTINGS
\topmargin=-0.45in
\evensidemargin=0in
\oddsidemargin=0in
\textwidth=6.5in
\textheight=9.0in
\headsep=0.25in
\linespread{1.1}
\pagestyle{fancy}
\lhead{\hmwkAuthorName}
\chead{}
% \rhead{\hmwkStudentnum}
\rhead{\MakeLowercase{\leftmark}}
\lfoot{\lastxmark}
\cfoot{\thepage}
\renewcommand\headrulewidth{0.4pt}
\renewcommand\footrulewidth{0.4pt}
\setlength\parindent{0pt}


\renewcommand{\part}[1]{\textbf{\large Part \Alph{partCounter}}\stepcounter{partCounter}\\} % part macro
\newcommand{\solution}{\textbf{\large Solution}} % solution macro

\newmdenv[
    backgroundcolor=gray!20,
    skipabove=\topsep,
    skipbelow=\topsep,
]{grayBoxed}

% General writing
\newcommand{\bracket}[1]{\left(#1\right)} % for automatic resizing of brackets
\newcommand{\sbracket}[1]{\left[#1\right]} % for automatic resizing of square brackets
\newcommand{\mset}[1]{\left\{#1\right\}} % for automatic resizing of curly brackets
\newcommand{\defeq}{\coloneqq} % the "defined as" command
\newcommand{\RNum}[1]{\uppercase\expandafter{\romannumeral #1\relax}} % uppercase roman numerals

\newcommand{\rednote}[1]{{\color{red} #1}} % for red text
\newcommand{\bluenote}[1]{{\color{blue} #1}} % for blue text
\newcommand{\greennote}[1]{{\color{green} #1}} % for green text

% Blackboard Maths Symbols
\newcommand{\N}{\mathbb{N}} % natural numbers
\newcommand{\Q}{\mathbb{Q}} % rational numbers
\newcommand{\Z}{\mathbb{Z}} % integers
\newcommand{\R}{\mathbb{R}} % real numbers
\newcommand{\C}{\mathbb{C}} % complex numbers
\newcommand{\E}{\mathbb{E}} % expectation operators

% Aesthetic
\newcommand{\thmBox}[2]{
    \begin{tcolorbox}[colback=blue!5!white,colframe=blue!50!black,
            colbacktitle=blue!90!black,title=Theorem #1]
        #2
    \end{tcolorbox}
}

\newcommand{\lmBox}[2]{
    \begin{tcolorbox}[colback=blue!5!white,colframe=blue!50!black,
            colbacktitle=blue!50!black,title=Lemma #1]
        #2
    \end{tcolorbox}
}

\newcommand{\defBox}[2]{
    \begin{tcolorbox}[colback=red!5!white,colframe=red!50!black,
            colbacktitle=red!60!black,title=Definition #1]
        #2
    \end{tcolorbox}
}


\newcommand{\exerciseBox}[2]{
    \begin{tcolorbox}[colback=red!5!white,colframe=red!50!black,
            colbacktitle=red!60!black,title=Definition #1]
        #2
    \end{tcolorbox}
}

% Homework Specific
\newcommand{\hBox}[1]{
    \begin{tcolorbox}[colback=blue!5!white,colframe=blue!50!black]
        #1
    \end{tcolorbox}
}

\newcommand{\hmwkStudentnum}{21319203}
\newcommand{\hmwkAuthorName}{\textbf{Angel Cervera Roldan}}

%%% Define the homeworkProblem environment
\newcommand{\enterProblemHeader}[1]{
    \nobreak\extramarks{}{Problem \arabic{#1} continued on next page\ldots}\nobreak{}
    \nobreak\extramarks{Problem \arabic{#1} (continued)}{Problem \arabic{#1} continued on next page\ldots}\nobreak{}
}

\newcommand{\exitProblemHeader}[1]{
    \nobreak\extramarks{Problem \arabic{#1} (continued)}{Problem \arabic{#1} continued on next page\ldots}\nobreak{}
    \stepcounter{#1}
    \nobreak\extramarks{Problem \arabic{#1}}{}\nobreak{}
}

\setcounter{secnumdepth}{0}
\newcounter{partCounter}
\newcounter{homeworkProblemCounter}
\setcounter{homeworkProblemCounter}{1}
\nobreak\extramarks{Problem \arabic{homeworkProblemCounter}}{}\nobreak{}

\newenvironment{homeworkProblem}[1][-1]{
    \ifnum#1>0
        \setcounter{homeworkProblemCounter}{#1}
    \fi
    \section{Problem \arabic{homeworkProblemCounter}}
    \setcounter{partCounter}{1}
    \enterProblemHeader{homeworkProblemCounter}
}{
    \exitProblemHeader{homeworkProblemCounter}
}
